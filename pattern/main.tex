%%% Если работа выполняется в Overleaf, 
%%% важно поменять тип компилятора
%%% с "pdfLatex" или "Latex" на "XeLaTeX". 
%%% Если этого не сделать, то и и компиляция
%%% не будет работать, и при загрузке текста ВКР
%%% на сайт МФТИ будут проблемы. 

%%% Если диплом получается очень маленький, 
%%% смело увеличивай шрифт до 14го
\documentclass[xetex,a4paper,12pt]{article} % 12й шрифт

% Подключение преамбулы с необходимыми пакетами
%%% Преамбула %%%

% Размер текста
%%%\usepackage[fontsize=14]{fontsize}

\usepackage{fontspec} % XeTeX
\usepackage{xunicode} % Unicode для XeTeX
\usepackage{xltxtra}  % Верхние и нижние индексы
\usepackage{pdfpages} % Вставка PDF

\usepackage{listings} % Оформление исходного кода
\lstset{
    basicstyle=\small\ttfamily, % Размер и тип шрифта
    breaklines=true,            % Перенос строк
    tabsize=2,                  % Размер табуляции
    frame=single,               % Рамка
    literate={--}{{-{}-}}2,     % Корректно отображать двойной дефис
    literate={---}{{-{}-{}-}}3  % Корректно отображать тройной дефис
}

% Шрифты, xelatex
\defaultfontfeatures{Ligatures=TeX}
\setmainfont{Times New Roman} % Нормоконтроллеры хотят именно его
\newfontfamily\cyrillicfont{Times New Roman}
\setmonofont{FreeMono} % Моноширинный шрифт для оформления кода

% Формулы
\usepackage{mathtools} % Не совместим с amsmath
\usepackage[warnings-off={mathtools-colon,mathtools-overbracket}]{unicode-math}
\setmathfont{XITS Math}             % Шрифт для формул: https://github.com/khaledhosny/xits-math
\numberwithin{equation}{section}    % Формула вида секция.номер

% Дерево директорий
\usepackage{forest}
\forestset{
  default preamble={
    for tree={
        grow'=0,
        child anchor=west,
        parent anchor=south,
        anchor=west,
        calign=first,
        edge path={
          \noexpand\path [draw, \forestoption{edge}]
          (!u.south west) +(7.5pt,0) |- node[fill,inner sep=1.25pt] {} (.child anchor)\forestoption{edge label};
        },
        before typesetting nodes={
          if n=1
            {insert before={[,phantom]}}
            {}
        },
        fit=band,
        before computing xy={l=15pt},
      }
  }
}

% Русский язык
\usepackage{polyglossia}
\setdefaultlanguage{russian}

% Абзацы и списки
\usepackage{enumerate}   % Тонкая настройка списков
\usepackage{indentfirst} % Красная строка после заголовка
\usepackage{float}       % Расширенное управление плавающими объектами
\usepackage{multirow}    % Сложные таблицы

% Пути к каталогам с изображениями
\usepackage{graphicx} % Вставка картинок и дополнений
\graphicspath{{img/}}

% Формат подрисуночных записей
\usepackage{chngcntr}

% Сбрасываем счетчик таблиц и рисунков в каждой новой главе
\counterwithin{figure}{section}
\counterwithin{table}{section}

% Гиперссылки
\usepackage{hyperref}
\hypersetup{
    colorlinks, urlcolor={black}, % Все ссылки черного цвета, кликабельные
    linkcolor={black}, citecolor={black}, filecolor={black},
    pdfauthor={Иван Чудинов},
    pdftitle={Разработка модели артериального тромбоза на основе гидрогеля}
}

% Оформление библиографии и подрисуночных записей через точку
\makeatletter
\renewcommand*{\@biblabel}[1]{\hfill#1.}
\renewcommand*\l@section{\@dottedtocline{1}{1em}{1em}}
%%% Здесь может начать бухтеть и давать предупреждения, пока не найдет в тексте хотя бы один рисунок и/или хотя бы одну таблицу. Если это бесит, то закомментируй. 
\renewcommand{\thefigure}{\thesection.\arabic{figure}} % Формат рисунка секция.номер
\renewcommand{\thetable}{\thesection.\arabic{table}}   % Формат таблицы секция.номер
\def\redeflsection{\def\l@section{\@dottedtocline{1}{0em}{10em}}}
\makeatother

\renewcommand{\baselinestretch}{1.4} % Полуторный межстрочный интервал
\parindent 1.27cm % Абзацный отступ

\sloppy             % Избавляемся от переполнений
\hyphenpenalty=1000 % Частота переносов
\clubpenalty=10000  % Запрещаем разрыв страницы после первой строки абзаца
\widowpenalty=10000 % Запрещаем разрыв страницы после последней строки абзаца

% Отступы у страниц
\usepackage[left=3cm,right=1.5cm,bottom=2cm,top=2cm]{geometry}

% Списки
\usepackage{enumitem}
\setlist[enumerate,itemize]{leftmargin=12.7mm} % Отступы в списках

\makeatletter
    \AddEnumerateCounter{\asbuk}{\@asbuk}{м)}
\makeatother
\setlist{nolistsep}                           % Нет отступов между пунктами списка
\renewcommand{\labelitemi}{--}                % Маркер списка --
\renewcommand{\labelenumi}{\asbuk{enumi})}    % Список второго уровня
\renewcommand{\labelenumii}{\arabic{enumii})} % Список третьего уровня

% Содержание
\usepackage{tocloft}
\renewcommand{\cfttoctitlefont}{\hspace{0.38\textwidth}\MakeTextUppercase} % СОДЕРЖАНИЕ
\renewcommand{\cftsecfont}{\hspace{0pt}}            % Имена секций в содержании не жирным шрифтом
\renewcommand\cftsecleader{\cftdotfill{\cftdotsep}} % Точки для секций в содержании
\renewcommand\cftsecpagefont{\mdseries}             % Номера страниц не жирные
\setcounter{tocdepth}{3}                            % Глубина оглавления, до subsubsection

% Список иллюстративного материала
\renewcommand{\cftloftitlefont}{\hspace{0.17\textwidth}\MakeTextUppercase}
\renewcommand{\cftfigfont}{Рисунок }
\addto\captionsrussian{\renewcommand\listfigurename{Список иллюстративного материала}}

% Список табличного материала
\renewcommand{\cftlottitlefont}{\hspace{0.2\textwidth}\MakeTextUppercase}
\renewcommand{\cfttabfont}{Таблица }
\addto\captionsrussian{\renewcommand\listtablename{Список табличного материала}}

% Нумерация страниц посередине снизу
\usepackage{fancyhdr}
\pagestyle{fancy}
\fancyhf{}
\cfoot{\textrm{\thepage}}
\fancyheadoffset{0mm}
\fancyfootoffset{0mm}
\setlength{\headheight}{17pt}
\renewcommand{\headrulewidth}{0pt}
\renewcommand{\footrulewidth}{0pt}
\fancypagestyle{plain}{
    \fancyhf{}
    \cfoot{\textrm{\thepage}}
}

% Формат подрисуночных надписей
\RequirePackage{caption}
\DeclareCaptionLabelSeparator{defffis}{ -- } % Разделитель
\captionsetup[figure]{justification=centering, labelsep=defffis, format=plain} % Подпись рисунка по центру

\captionsetup[table]{justification=raggedright, labelsep=defffis, format=plain, singlelinecheck=false} % Подпись таблицы слева
\addto\captionsrussian{\renewcommand{\figurename}{Рисунок}} % Имя фигуры

% Пользовательские функции
\newcommand{\addimg}[4]{ % Добавление одного рисунка
    \begin{figure}
        \centering
        \includegraphics[width=#2\linewidth]{#1}
        \caption{#3} \label{#4}
    \end{figure}
}
\newcommand{\addimghere}[4]{ % Добавить рисунок непосредственно в это место
    \begin{figure}[H]
        \centering
        \includegraphics[width=#2\linewidth]{#1}
        \caption{#3} \label{#4}
    \end{figure}
}
\newcommand{\addtwoimghere}[6]{ % Вставка двух рисунков
    \begin{figure}[H]
        \centering
        \includegraphics[width=#3\linewidth]{#1}
        \hfill
        \includegraphics[width=#4\linewidth]{#2}
        \caption{#5} \label{#6}
    \end{figure}
}

\usepackage{wasysym} % Пришлось добавить для символов из статьи
\usepackage{placeins} % Для FloatBarrier

% Заголовки секций в оглавлении в верхнем регистре
\usepackage{textcase}
\makeatletter
\let\oldcontentsline\contentsline
\def\contentsline#1#2{
    \expandafter\ifx\csname l@#1\endcsname\l@section
        \expandafter\@firstoftwo
    \else
        \expandafter\@secondoftwo
    \fi
    {\oldcontentsline{#1}{\MakeTextUppercase{#2}}}
    {\oldcontentsline{#1}{#2}}
}
\makeatother

% Оформление заголовков
\usepackage[compact,explicit]{titlesec}
\titleformat{\section}{}{}{12.5mm}{\centering{\thesection\quad\MakeTextUppercase{#1}}\vspace{1.5em}}
\titleformat{\subsection}[block]{\vspace{1em}}{}{12.5mm}{\thesubsection\quad#1\vspace{1em}}
\titleformat{\subsubsection}[block]{\vspace{1em}\normalsize}{}{12.5mm}{\thesubsubsection\quad#1\vspace{1em}}
\titleformat{\paragraph}[block]{\normalsize}{}{12.5mm}{\MakeTextUppercase{#1}}

% Секции без номеров (введение, заключение...), вместо section*{}
\newcommand{\anonsection}[1]{
    \phantomsection % Корректный переход по ссылкам в содержании
    \paragraph{\centerline{{#1}}\vspace{1em}}
    \addcontentsline{toc}{section}{#1}
}

% Сабсекции без номеров, вместо subsection*{}
\newcommand{\anonsubsection}[1]{
    \vspace{1em}
}

% Секция для аннотации (она не включается в содержание)
\newcommand{\annotation}[1]{
    \paragraph{\centerline{{#1}}\vspace{1em}}
}

% Секция для списка иллюстративного материала
\newcommand{\lof}{
    \phantomsection
    \listoffigures
    \addcontentsline{toc}{section}{\listfigurename}
}

% Секция для списка табличного материала
\newcommand{\lot}{
    \phantomsection
    \listoftables
    \addcontentsline{toc}{section}{\listtablename}
}

% Секции для приложений
\newcommand{\appsection}[1]{
    \phantomsection
    \paragraph{\centerline{{#1}}}
    \addcontentsline{toc}{section}{{#1}}
}

% Библиография: отступы и межстрочный интервал
\makeatletter
\renewenvironment{thebibliography}[1]
    {\section*{\refname}
        \list{\@biblabel{\@arabic\c@enumiv}}
           {\settowidth\labelwidth{\@biblabel{#1}}
            \leftmargin\labelsep
            \itemindent 16.7mm
            \@openbib@code
            \usecounter{enumiv}
            \let\p@enumiv\@empty
            \renewcommand\theenumiv{\@arabic\c@enumiv}
        }
        \setlength{\itemsep}{0pt}
    }
\makeatother

\usepackage{lastpage} % Подсчет количества страниц
\setcounter{page}{1}  % Начало нумерации страниц

\begin{document}
    %%% При загрузке текста ВКР на сайт МФТИ титульная страница
    %%% будет сгенерирована автоматически, 
    %%% скачиваешь её, добавляешь в папку inc. 
    %%% Когда будет готово, включишь эту строчку. 
    %%% \includepdf[]{inc/0-titlepage.pdf} % Титульная страница
    
    \annotation{Аннотация}

Тема выпускной квалификационной работы бакалавра --- <<Сравнительный анализ монолитного и микросервисного подходов к разработке web приложений в условиях низкой загрузки>>.

В данной работе проводится сравнительный анализ монолитного и микросервисного подходов к разработке веб-приложений в условиях низкой загрузки. Для исследования выбраны два примера проектов: монолитная структура, представленная проектом о медицинских перевозках, и микросервисная архитектура, представленная проектом о каталогизации данных производств.

Целью работы является выявление преимуществ и недостатков каждого подхода, их эффективности и применимости в контексте веб-приложений с низкой загрузкой. В работе осуществляется анализ предметной области, изучение теоретических аспектов монолитной и микросервисной архитектур, а также описание процесса разработки двух проектов, на основе которых проводится сравнение ключевых характеристик архитектур, и заключение о перспективах применения каждого подхода.

В результате работы выясняется, что в условиях низкой загрузки монолитная архитектура имеет преимущество ввиду своей простоты, логической целостности и удобства. Исследования представляют собой важный вклад в понимание особенностей различных архитектурных подходов и помогут разработчикам и архитекторам в выборе наиболее подходящего решения при проектировании собственных систем.

% Обязательно добавляем это в конце каждой секции, чтобы 
% обеспечить переход на новую страницу
\clearpage % Аннотация
    \tableofcontents % Содержание 
    \clearpage

    \anonsection{Введение}
    TODO
    В современном мире веб-разработка стала неотъемлемой частью программирования. Согласно статистике Гитхаб \cite{githut}, более 15\% звёзд принадлежат репозиториям, использующим язык Java Script, что однозначно показывает популярность веб разработки. При этом, согласно статистике сайта siteefy \cite{websitesstats}, более 250000 сайтов создаются каждый день - следовательно, популярна не только поддержка старых репозиториев веб разработки, но и разработка новых сайтов.
    
    Выбор архитектурного подхода при создании веб-приложений играет решающую роль в обеспечении их эффективности, масштабируемости и удобства использования. Не случайно вопрос выбора между монолитной и микросервисной архитектурами остается одним из наиболее актуальных в сфере веб-разработки. Выбор архитектурного подхода определяет не только технические характеристики разрабатываемого приложения, но и его конечное качество и степень соответствия потребностям пользователей; это связано с тем, что архитектура веб-приложения влияет на его производительность, масштабируемость, надежность, безопасность, и многие другие аспекты, определяющие его успешность на рынке.
    
    Несмотря на появление новых подходов и технологий, выбор между монолитной и микросервисной архитектурами всё ещё остается актуальным. Монолитная архитектура - традиционный способ построения приложения, при котором весь функционал собирается в одном приложении или модуле, и все компоненты, функции, классы взаимодействуют друг с другом напрямую а рамках одного процесса. Микросервисная архитектура, в противовес монолитной - современный подход, использующий разделение логически разных частей приложения на физически разные компоненты - микросервисы. Каждый микросервис разрабатывается и разворачивается независимо, общаются между собой чаще всего через http запросы. Монолитные приложения сохраняют свою популярность благодаря своей простоте и относительной легкости в развертывании, в то время как микросервисные архитектуры привлекают внимание своей гибкостью и масштабируемостью.
    
    В данной работе мы сфокусируемся на  на анализе нефункциональных требований в контексте выбора архитектуры для веб-приложений. В частности, мы рассмотрим влияние выбора между монолитной и микросервисной архитектурами на безопасность, производительность, масштабируемость и другие аспекты разработки веб-приложений.

% Обязательно добавляем это в конце каждой секции, чтобы 
% обеспечить переход на новую страницу
\clearpage % Введение
    \anonsection{Основные понятия}

    \subsection*{Java}
        Java - строго типизированный объектно-ориентированный язык программирования общего назначения, разработанный компанией Sun Microsystems. Разработка ведётся сообществом, организованным через Java Community Process; язык и основные реализующие его технологии распространяются по лицензии GPL. Приложения Java обычно транслируются в специальный байт-код, поэтому они могут работать на любой компьютерной архитектуре, для которой существует реализация виртуальной Java-машины. Дата официального выпуска — 23 мая 1995 года. Занимает высокие места в рейтингах популярности языков программирования (2-е место в рейтингах IEEE Spectrum (2020) и TIOBE (2021)).

    \subsection*{Spring Framework}
        Spring Framework — универсальный фреймворк с открытым исходным кодом для Java-платформы. Фреймворк был впервые выпущен под лицензией Apache 2.0 license в июне 2003 года. Первая стабильная версия 1.0 была выпущена в марте 2004. Spring 2.0 был выпущен в октябре 2006, Spring 2.5 — в ноябре 2007, Spring 3.0 в декабре 2009, и Spring 3.1 в декабре 2011. Текущая версия — 5.3.x. Несмотря на то, что Spring не обеспечивал какую-либо конкретную модель программирования, он стал широко распространённым в Java-сообществе главным образом как альтернатива и замена модели Enterprise JavaBeans. Spring предоставляет бо́льшую свободу Java-разработчикам в проектировании; кроме того, он предоставляет хорошо документированные и лёгкие в использовании средства решения проблем, возникающих при создании приложений корпоративного масштаба.

    \subsection*{Keycloak}
        Keycloak - продукт с открытым кодом для реализации single sign-on с возможностью управления доступом, нацелен на современные приложения и сервисы. По состоянию на 2018 год, этот проект сообщества JBoss находится под управлением Red Hat которые используют его как upstream проект для своего продукта RH-SSO. Целью этого инструмента является сделать создание безопасных приложений и сервисов с минимальным написанием кода для аутентификации и авторизации. 

    \subsection*{Spring Cloud Netflix Eureka}
        Spring Cloud Netflix Eureka - инструмент для обнаружения служб на стороне клиента. Позволяет службам находить друг друга и связываться друг с другом без жесткого кодирования имени хоста и порта. Единственной «фиксированной точкой» в такой архитектуре является реестр служб, в котором должна регистрироваться каждая служба.

    \subsection*{Микросервисная архитектура}
        Микросервисная архитектура предполагает разработку и поддержку приложений с использованием небольших модульных сервисов, а не создание программного обеспечения в виде одного большого унифицированного блока кода (монолита). Основная концепция архитектуры в том, чтобы разделить сложное приложение на несколько небольших автономных и управляемых компонентов. Это позволяет повысить гибкость разработки, сократить time-to-market, улучшить отказоустойчивость и облегчить поддержку приложения.
        Каждый микросервис имеет собственный набор кода, базу данных и API для взаимодействия с другими сервисами. Они могут быть написаны на разных языках программирования и использовать различные технологии. Взаимодействие сервисов может осуществляться посредством сетевых запросов или сообщений.

    \subsection*{Монолитная архитектура архитектура}
        Монолитная архитектура — это традиционная модель программного обеспечения, которая представляет собой единый модуль, работающий автономно и независимо от других приложений. Монолитом часто называют нечто большое и неповоротливое, и эти два слова хорошо описывают монолитную архитектуру для проектирования ПО. Монолитная архитектура — это отдельная большая вычислительная сеть с единой базой кода, в которой объединены все бизнес-задачи.

% Обязательно добавляем это в конце каждой секции, чтобы 
% обеспечить переход на новую страницу
\clearpage % Основные понятия
    \section{Анализ предметной области}

\subsection{Описание проблемы}
    //TODO
    https://colorlib.com/wp/web-development-statistics/

\subsection{Описание основных принципов и особенностей каждого подхода}
        Монолитный подход:
        Единый исполняемый контейнер: все компоненты приложения размещаются в одном исполняемом контейнере.
        Монолитные приложения обычно написаны на одном языке программирования и используют одну базу данных.
        Простота развертывания и управления: так как все компоненты находятся в одном приложении, развертывание и управление приложением обычно проще.
        Ограничения в масштабировании: монолитные приложения могут столкнуться с проблемами масштабирования при увеличении объема функциональности.

    Микросервисный подход:
        Разделение на небольшие сервисы: приложение разбивается на независимые компоненты, каждый из которых выполняет определенную функциональность.
        Гибкость в выборе технологий: каждый сервис может быть написан на разных языках программирования и использовать различные технологии.
        Распределенная архитектура: каждый сервис может иметь свою собственную базу данных и взаимодействует с другими сервисами через API.
        Масштабируемость и гибкость: микросервисы позволяют гибко масштабировать отдельные компоненты приложения в зависимости от нагрузки.
        
    Оба подхода имеют свои преимущества и недостатки, и выбор между ними зависит от конкретных требований проекта, его масштаба и потребностей в гибкости и масштабируемости.

    Преимущества и недостатки монолитной и микросервисной архитектур:

    Монолитная архитектура:
    
    Преимущества:
    
        Простота разработки: единое приложение облегчает разработку и тестирование.
        Проще в управлении: единый контейнер упрощает развертывание и мониторинг приложения.
        Меньшие накладные расходы: отсутствие сетевых запросов между компонентами приложения снижает накладные расходы на коммуникацию.
    
    Недостатки:
    
        Сложности в масштабировании: монолитные приложения могут столкнуться с проблемами масштабирования при увеличении объема функциональности.
        Затруднения в обновлении: изменения в одной части приложения могут повлиять на всю систему и требовать полного пересборки и перезапуска.
        Ограниченная гибкость: сложнее добавлять новые функции и технологии из-за зависимостей между компонентами.
    
    Микросервисная архитектура:
    
    Преимущества:
    
        Гибкость и масштабируемость: каждый сервис может масштабироваться независимо, что обеспечивает гибкость в управлении ресурсами.
        Улучшенная отказоустойчивость: отказ одного сервиса не приводит к полной недоступности приложения.
        Технологическая свобода: различные сервисы могут быть реализованы на разных технологиях, что позволяет использовать наиболее подходящие инструменты для каждой задачи.
    
    Недостатки:
    
        Сложности в развертывании и управлении: управление распределенными системами требует дополнительных усилий и навыков.
        Увеличение сложности тестирования: необходимость тестирования взаимодействия между сервисами увеличивает сложность тестирования.
        Дополнительные затраты на сетевое взаимодействие: сетевые запросы между сервисами могут вызывать задержки и увеличивать нагрузку на сеть.


\subsection{Аналогичные исследования}
    //TODO
    
        
% Обязательно добавляем это в конце каждой секции, чтобы 
% обеспечить переход на новую страницу
\clearpage % Анализ предметной области
    \section{Реализация}

    В данном разделе будут приведены описания два приложения, разработанные и реализованные под моим руководством. Будет рассказано про основные требования, выдвинутые к данным проектам, их структура, команды для работы с ними, описаны инструменты разработки и сервера, на которых проихводился деплой. Всё это будет использовано в следующих разделах в сравнительном анализе архитектур, чтобы сравнение происходило на прикладных примерах из практики.

\subsection{Монолит - Medmobil}

    \subsubsection{Общее описание}
    Разработка этого проекта началась с нуля. К нашей команде обратился заказчик с предложением разработать сайт для организации медицинских перевозок. Изначально данный проект представлялся в виде разового проекта, хотя после вывода MVP разработка продолжалась в качестве постоянной работы. Было необходимо реализовать сервис, который бы позволял осуществлять:
    \begin{itemize}
        \item просмотр нескольких информационных страниц;
        \item регистрация и авторизация пользователей, включая ролевую модель - обычные пользователи, владельцы услуг, их диспетчеры и администраторы;
        \item возможность создания заказа с многочисленными различными параметрами;
        \item возможность управления ролевой моделью в рамках полномочий роли авторизованного пользователя;
        \item возможность просмотра, отслеживания и редактирования заказов в зависимости от роли авторизованного пользователя.
    \end{itemize}
    Целью было создать аналог сервиса Яндекс.Go, но только для заказа медицинского транспорта для первозки больных в различном состоянии, как запланированных, так и экстренных. При этом спрос на медицинский транспорт оценивается сверху 100 заказами в день, поэтому данный проект был реализован под низкую загруженность.

    \subsubsection{План разработки}
    Первоначальная идея представляла собой следующий план действий:
    \begin{itemize}
        \item разработка теоретической модели базы данных;
        \item разработка User Stories для основных сценариев на основе представлений заказчика;
        \item создание базовой части проекта (инициализация проектов PSQL, Java, Vue и их связь);
        \item реализация MVP (создание моделей в базе данных, разработка бэкенда на Java и фронтенда на Vue);
        \item аренда сервера и деплой с открытым IP.
    \end{itemize}
    Помимо этого было принято решение использовать Github репозиторий для хранения кодовой базы, а также для проведения автоматических тестов и проверки кодстайла.
    По завершению разработки MVP был сформирован новый план, направленный на выпуск приложения в открытый мир:
    \begin{itemize}
        \item подключение авторизации и регистрации через сторонний сервис (Keycloak);
        \item снижение задержки между обнаружением ошибки в поведении программе и её ликвидацией;
        \item снижение ежемесячных трат на поддержание стендов для теста и прода;
        \item оптимизация скорости работы (в основном направлено на часть фронтенда).
    \end{itemize}

    \subsubsection{Требования}
    Требования к проекту менялись в соответствии с планами и уточнялись в процессе разработки. В начале приоритеты были направлены на сокращение времени и оптимизацию "фундамента" для будущей разработки. Поэтому, к нашей команде были выдвинуты следующие требования: 
    \begin{itemize}
        \item скорость разработки;
        \item разработка расширяемой архитектуры;
        \item применении принципов разработки (DRY, YAGNI, SOLID) с целью создания удобочитаемого, поддерживаемого кода.
    \end{itemize}
    После создания кодовой базы и вывода первоначальной версии на стенд прода, требования были изменены в сторону поддержки уже существующей кодовой базы:
    \begin{itemize}
        \item повышение совместимости для упрощения совместимости с внешними сервисами (Keycloak);
        \item повышение наблюдаемости для увеличения скорости нахождения и исправления ошибок;
        \item уменьшение потребления ресурсов.
    \end{itemize}

    \subsubsection{Архитектура}
    Выбор архитектуры происходил в соответствии с требованиями, представленными выше. Поэтому была выбрана монолитная архитектура (подробнее про причины выбора будет рассказано в разеделе Анализ). Итоговая структура репозитория выглядела так:
    \newline \newline
    \begin{forest}
    [
      [src/
        [main/
          [...]
        ]
        [test/
          [...]
        ]
      ]
      [vueMedicalService/
        [package.json]
        [vue.config.js]
        [src/
          [...]
        ]
      ]
      [target/
        [...]
      ]
      [codestyle/
        [...]
      ]
    ]
    \end{forest}
    
    Как видно, все файлы лежат в одном репозитории. Папка src/ содержит все файлы, относящиеся к бэкенду - проекту на Java. Папка vueMedicalService содержит все файлы, относящиеся к фронтенду - проекту на Vue.js. Также есть общие файлы, которые относятся к проекту в целом - папка target/ содержит итоговые собранные файлы, codestyle/ содержит скрипты и конфигурационные файлы для проверки стиля кода.



    \subsection{Микросервисы}
    
    \subsubsection{Общее описание}
    Разработка данного проекта началась практически с нуля. Единственное, что было дано нашей команде - это две заготовки под микросервисы, которые не были объединены никоим образом и содержали лишь базовую логику GET/SET для простой таблице в базе данных. К нашей команде обратился заказчик с предложением разработать сайт для каталогизации и структуризации данных различных производств. Идея заключается в создании единого каталога для всевозможных технологических процессов, материалов и деталей, которые соединяются в единую производственную цепочку в соответствии с параметрами и характеристиками данной сущности.

    Было необходимо реализовать сервис, который бы позволял осуществлять:
    \begin{itemize}
      \item просмотр нескольких информационных страниц;
      \item регистрация и авторизация пользователей, включая ролевую модель - обычные пользователи и администраторы;
      \item возможность управления ролевой моделью в рамках полномочий роли авторизованного пользователя; 
      \item возможность добавления новых сущностей и параметров (в случае наличия достаточных прав);
      \item возможность просмотра, поиска и отображения в GUI (в табличном виде, а объектов, имеющих географическое положение, в том числе и на карте) данных о сущностях.
    \end{itemize}

    \subsubsection{План разработки}
    План разработки изначально был более детализированный и продуманный на больший срок, поскольку было необходимо разрабатывать платформу для будущей тяжеловесной системы, анализирующей большие данные. По этой же причине уже заказчиком была выбрана микросервисная архитектура. Тем не менее, мы вели разработку в условиях низкой загруженности, поскольку разработкой анализа больших данных должна была заниматься другая команда, после окончания нашей рзработки системы в целом. План разработки:
    \begin{itemize}
        \item разработка теоретической модели базы данных с учётом специфики проекта (неопределённое количество параметров каждой сущности, необходимость рекурсивно вложенных отношений);
        \item разработка User Stories для основных сценариев на основе представлений заказчика;
        \item разработка микросервисов, необходимых для поддержания базы данных и операций, необходимых для корректного манипулирования данными
        \item создание базовой части проекта (инициализация проектов PSQL, Java, Vue);
        \item разработка и создание сервисов, необходимых для корректной связи микросервисов (Eureka, API Gateway, Config)
        \item подключение авторизации и регистрации через сторонний сервис (Keycloak);
        \item реализация MVP (создание моделей в базе данных, разработка бэкенда на Java и фронтенда на Vue);
        \item аренда сервера и деплой с открытым IP.
    \end{itemize}
    Также было принято решение использовать Gitlab репозиторий для хранения кодовой базы, а также для проведения автоматических тестов и проверки кодстайла.

    \subsubsection{Требования}
    Требования к проекту формировались в соответствии с планами. Были сразу поставлены приоритеты на проектирование системы с ориентиром на последующую доработку проекта в высоконагруженный сервис. Поэтому, к нашей команде были выдвинуты следующие требования: 
    \begin{itemize}
        \item разработка расширяемой архитектуры;
        \item применении принципов разработки (DRY, YAGNI, SOLID) с целью создания удобочитаемого, поддерживаемого кода;
        \item обеспечение совместимости кода для интеграции с внешними сервисами (Keycloak);
        \item повышение наблюдаемости для увеличения скорости нахождения и исправления ошибок;
        \item уменьшение потребления ресурсов;
        \item возможность масштабирования.
    \end{itemize}

    \subsubsection{Архитектура}
    Для данного проекта была выбрана микросервисная архитектура (подробнее про причины выбора будет рассказано в разеделе Анализ). Итоговая структура репозитория выглядела так:
    \newline \newline
    \begin{forest}
    [
      [java/
        [docker-compose.yml]
        [docker/
          [Dockerfile]
          [Dockerfile-keycloak]
        ]
        [utils/
          [\{java\_project\}]
        ]
        [eureka/
          [\{java\_project\}]
        ]
        [gateway/
          [\{java\_project\}]
        ]
        [microservices/
          [amqp-service/
            [\{java\_project\}]
          ]
          [company-data/
            [\{java\_project\}]
          ]
          [import-data/
            [\{java\_project\}]
          ]
          [master-data/
            [\{java\_project\}]
          ]
          [service-auth/
            [\{java\_project\}]
          ]
        ]
      ]
      [vue/
        [...]
      ]
    ]
    \end{forest}
    \newline
    где сокращение [java\_project] означает стандартный Java проект: pom.xml и папка src/, в которой хранится вся кодовая база.
    
    Видно, что существует множество различных папок в бэкенде на Java, так как появилось множество отдельных подпроектов, которые отвечают за микросервисы. Также есть общие файлы, которые относятся к проекту в целом - папка utils/ содержит общие классы, docker/ содержит файлы для docker-compose.



    \subsection{Корректность}
    Как видно из приведённого выше описания, данные проекты очень похожи друг на друга: оба были целиком созданы мной и моей командой, используется одинаковый стэк. План разработки похож, хотя во втором проекте и присутствует больше шагов ввиду усложнённой структуры проекта с микросервисной архитектурой (подробнее будет в разделе Анализ). Архитектура проектов не имеет принципиальных дополнений по сравнению с базовым проектом Java, но заметно, что во втором проекте больше файлов и папок.



% Обязательно добавляем это в конце каждой секции, чтобы 
% обеспечить переход на новую страницу
\clearpage % Реальные проекты
    \section{Сравнение}

\subsection{Виды критериев}
    В целом, сравнения двух приложений и определения преимущества того или иного подхода (монолитного или микросервисного) можно учитывать следующие параметры:

    \begin{itemize}
        \item Масштабирование
        \item Производительность
        \item Отказоустойчивость
        \item Надёжность
        \item Переиспользуемость
        \item Команда
        \item Скорость разработки
        \item Совместимость
        \item Переносимость
        \item Наблюдаемость
        \item Потребление ресурсов
    \end{itemize}

    Данные параметры были взяты из исследований: "...возможность быстро обнаружить неполадки..." \cite{шитько2017проектирование} (т. е. наблюдаемость) и "...обеспечить отказоустойчивость и балансировку нагрузки..." и "...требуется больше вычислительных ресурсов для функционирования множества сервисов..." \cite{артамонов2016разработка} (отказоустойчивость и потребление ресурсов). Также они были взяты из требований (описанных в разделе Реализация) и из пркатической разработки.

    Тем не менее, не все данные критерии применимы в ситуации








Функциональные требования:

    Функциональность приложения: Сравнение функциональных возможностей и способности приложений решать задачи и удовлетворять потребности пользователей.
    Гибкость и масштабируемость: Оценка возможности приложений адаптироваться к изменяющимся требованиям и масштабироваться для работы с различными объемами данных и нагрузками.
    Изоляция компонентов: Учитывание степени изоляции компонентов приложения и их независимости друг от друга для обеспечения гибкости разработки и поддержки.
    Развитие и поддержка: Оценка удобства разработки новых функций и поддержки существующих в рамках выбранной архитектуры.
    Интеграция с внешними системами: Учет возможности интеграции приложений с другими системами и сервисами для обеспечения расширенных функциональных возможностей.

Нефункциональные требования:

    Производительность: Сравнение производительности приложений в условиях различной нагрузки и объема данных.
    Отказоустойчивость и надежность: Оценка способности приложений обрабатывать ошибки и отказы без серьезных последствий для работы системы в целом.
    Безопасность: Анализ мер безопасности и возможности обеспечения защиты данных и системы от внешних угроз.
    Скорость развертывания и обновления: Оценка скорости и удобства развертывания новых версий приложений и обновлений.
    Затраты на разработку и поддержку: Учет затрат на разработку, развертывание и поддержку приложений в течение их жизненного цикла.

Учитывая эти параметры, можно определить, какой подход (монолитный или микросервисный) лучше соответствует требованиям и целям конкретного проекта.

Для объективной оценки и сравнения монолитного и микросервисного подходов по нефункциональным требованиям можно использовать следующие критерии:

    Безопасность:
        Монолитный подход: Аудит безопасности приложения в целом, включая проверку уязвимостей и обеспечение соответствия стандартам безопасности.
        Микросервисный подход: Анализ безопасности каждого сервиса отдельно, а также проверка наличия механизмов защиты данных при передаче между сервисами.

    Производительность:
        Монолитный подход: Оценка производительности единого приложения под нагрузкой, включая время отклика и скорость выполнения различных операций.
        Микросервисный подход: Оценка производительности каждого сервиса отдельно и их взаимодействия, а также оценка пропускной способности сети между сервисами.

    Масштабируемость:
        Монолитный подход: Оценка возможности масштабирования всего приложения целиком, включая возможность горизонтального и вертикального масштабирования.
        Микросервисный подход: Оценка гибкости и простоты масштабирования отдельных сервисов в зависимости от нагрузки на них, а также возможность автоматического масштабирования при необходимости.

    Доступность:
        Монолитный подход: Анализ уязвимых мест и резервных систем для обеспечения доступности приложения в случае отказов или сбоев.
        Микросервисный подход: Обеспечение высокой доступности путем дублирования и отказоустойчивости каждого сервиса, а также балансировки нагрузки между ними.

    Техническое обслуживание:
        Монолитный подход: Оценка сложности и затрат на обновление и поддержку всего приложения целиком, включая время, ресурсы и возможность автоматизации процессов.
        Микросервисный подход: Оценка сложности и затрат на обновление и поддержку каждого сервиса отдельно, а также возможность автоматизации и оркестрации процессов управления сервисами.

    Переносимость:
        Монолитный подход: Оценка возможности переноса приложения на различные платформы без необходимости крупных изменений в коде или архитектуре.
        Микросервисный подход: Оценка возможности переноса и совместимости каждого сервиса с различными платформами и средами выполнения.

    Надежность:
        Монолитный подход: Оценка стабильности и надежности работы всего приложения в целом, включая отказоустойчивость и восстановление после сбоев.
        Микросервисный подход: Оценка надежности каждого сервиса и их взаимодействия, а также механизмов обработки ошибок и восстановления после отказов.

    Юзабилити:
        Монолитный подход: Оценка удобства использования интерфейса и функциональности приложения для пользователей.
        Микросервисный подход: Анализ удобства использования и взаимодействия различных сервисов в рамках приложения для разработчиков и администраторов системы.

    Совместимость:
        Монолитный подход: Оценка совместимости приложения с другими системами и сервисами, с которыми оно взаимодействует.
        Микросервисный подход: Оценка совместимости каждого сервиса с другими системами и сервисами, а также возможность интеграции и обмена данными между ними.

    Юридические вопросы:
        Монолитный подход: Проверка соответствия всей архитектуры приложения применимым законам и правилам.
        Микросервисный подход: Анализ соответствия каждого сервиса применимым законам и правилам, а также возможность управления доступом к данным и сервисам с точки зрения правовых норм.

Анализ этих критериев для каждого из проектов позволит объективно оценить преимущества и недостатки монолитного и микросервисного подходов и выбрать наиболее подходящий вариант для конкретного случая.

\subsection{Безопасность}

Монолитный подход:

    Аудит безопасности приложения: Провести комплексный аудит безопасности приложения, включая анализ уязвимостей кода, проверку на соответствие стандартам безопасности и наличие механизмов защиты.
    Обеспечение безопасности данных: Оценить методы шифрования и защиты данных при хранении и передаче, а также наличие механизмов контроля доступа к конфиденциальной информации.

Микросервисный подход:

    Анализ безопасности каждого сервиса: Проверить каждый сервис на наличие уязвимостей и возможные точки входа для атак, а также оценить механизмы обеспечения безопасности внутри сервиса.
    Защита взаимодействия между сервисами: Обеспечить безопасность передачи данных между сервисами, включая использование протоколов шифрования и аутентификации.

Вывод:

    В монолитном подходе безопасность приложения оценивается как единое целое, что может упростить процесс аудита и обеспечения безопасности, но также повышает риск распространения уязвимостей на всю систему.
    В микросервисном подходе необходимо обеспечить безопасность каждого сервиса отдельно, что требует дополнительных усилий, но также обеспечивает более гибкий и гранулярный контроль над безопасностью в целом.

\subsection{Производительность}

Монолитный подход:

    Тестирование времени отклика: Измерить время отклика приложения на запросы пользователей при различных нагрузках для определения скорости работы системы.
    Анализ использования ресурсов: Оценить использование процессора, памяти и сети приложением во время работы для выявления узких мест и возможных улучшений.
    Скорость обработки запросов: Измерить время выполнения различных операций приложения, таких как загрузка страниц, выполнение запросов к базе данных и обработка данных.

Микросервисный подход:

    Тестирование производительности каждого сервиса: Измерить время отклика и производительность каждого сервиса отдельно при различных нагрузках для определения уровня его эффективности.
    Анализ времени передачи данных между сервисами: Оценить время передачи данных между сервисами и задержки в коммуникации для определения эффективности взаимодействия.
    Скорость обработки запросов и сборка данных: Измерить время выполнения запросов, необходимых для выполнения конкретной операции, и время, необходимое для сборки данных из разных сервисов.

Вывод:

    В монолитном подходе производительность приложения оценивается как единое целое, что может упростить процесс оптимизации, но также ограничивает возможности масштабирования и гибкости.
    В микросервисном подходе производительность каждого сервиса оценивается отдельно, что позволяет более гибко управлять ресурсами и масштабировать отдельные компоненты, но требует дополнительных усилий по управлению взаимодействием и коммуникацией между сервисами.

\subsection{Масштабируемость}

Монолитный подход:

    Горизонтальное масштабирование: Оценить возможность добавления дополнительных серверов для распределения нагрузки и увеличения пропускной способности приложения.
    Вертикальное масштабирование: Измерить возможность увеличения ресурсов на сервере (процессор, память) для обработки большего количества запросов и данных.

Микросервисный подход:

    Гибкость добавления новых сервисов: Оценить удобство и скорость добавления новых сервисов для расширения функциональности и увеличения масштабируемости.
    Гибкость масштабирования отдельных сервисов: Измерить возможность масштабирования отдельных сервисов по мере увеличения нагрузки на них.

Вывод:

    В монолитном подходе масштабируемость ограничена возможностями горизонтального и вертикального масштабирования всего приложения целиком, что может привести к ограничениям в управлении ресурсами и недостаточной гибкости.
    В микросервисном подходе масштабируемость повышается за счет возможности добавления и масштабирования отдельных сервисов по мере необходимости, что обеспечивает более гибкое управление ресурсами и адаптацию к изменяющимся требованиям.

\subsection{Доступность}

Монолитный подход:

    Тестирование отказоустойчивости: Провести тестирование на отказы и сбои приложения, включая проверку реакции системы на непредвиденные обстоятельства, такие как отключение серверов или сетевых узлов.
    Мониторинг доступности: Оценить доступность приложения в реальном времени с помощью мониторинговых инструментов, таких как системы мониторинга производительности и управления событиями.

Микросервисный подход:

    Тестирование отказоустойчивости каждого сервиса: Провести тестирование отказоустойчивости каждого сервиса отдельно, включая проверку реакции на сбои и ошибки.
    Балансировка нагрузки: Использовать механизмы балансировки нагрузки для обеспечения равномерного распределения нагрузки между сервисами и предотвращения перегрузок.

Вывод:

    В монолитном подходе доступность приложения оценивается как единое целое, что может упростить процесс управления и контроля, но также увеличивает риск целостного отказа системы при сбое.
    В микросервисном подходе каждый сервис оценивается отдельно по его доступности, что позволяет более точно контролировать и управлять отказами и сбоями, но также требует дополнительных усилий по управлению и мониторингу.

\subsection{Техническое обслуживание}

Монолитный подход:

    Сложность обновлений и патчей: Оценить сложность процесса обновления и установки патчей для всего приложения целиком, включая возможность автоматизации этих процессов.
    Общее время простоя: Измерить время простоя приложения во время обновлений и технического обслуживания для определения общего влияния на производственные процессы.

Микросервисный подход:

    Управление версиями сервисов: Оценить возможность управления версиями сервисов и их зависимостями, включая автоматизацию процессов обновления и деплоймента.
    Мониторинг здоровья сервисов: Использовать мониторинговые инструменты для отслеживания здоровья каждого сервиса и оперативного реагирования на сбои и проблемы.

Вывод:

    В монолитном подходе техническое обслуживание оценивается как единое целое, что может упростить процесс управления и контроля, но также увеличивает риск простоев системы при обновлениях.
    В микросервисном подходе управление и мониторинг сервисов может быть более гибким и эффективным, но также требует дополнительных усилий по организации и автоматизации процессов обслуживания.

\subsection{Переносимость}

Монолитный подход:

    Тестирование на разных платформах: Проверить работоспособность приложения на различных операционных системах и аппаратных платформах для определения степени переносимости.
    Использование кросс-платформенных технологий: Оценить использование кросс-платформенных технологий и инструментов разработки для обеспечения универсальности приложения.

Микросервисный подход:

    Независимость от платформы: Проверить, насколько каждый сервис является независимым от платформы, и как легко его можно перенести на другие среды выполнения.
    Стандартизация интерфейсов: Использовать стандартизированные протоколы и интерфейсы для обеспечения совместимости и переносимости сервисов.

Вывод:

    В монолитном подходе переносимость приложения оценивается как единое целое, что может упростить процесс тестирования и обеспечения совместимости, но также может ограничить возможности адаптации к различным средам выполнения.
    В микросервисном подходе каждый сервис оценивается отдельно по его переносимости, что позволяет более гибко управлять совместимостью и адаптацией к различным платформам, но также требует дополнительных усилий по обеспечению независимости и стандартизации.

\subsection{Надежность}

Монолитный подход:

    Тестирование отказоустойчивости: Провести тестирование на отказы и сбои приложения, включая проверку реакции системы на непредвиденные обстоятельства, такие как отключение серверов или сетевых узлов.
    Мониторинг и регистрация ошибок: Установить систему мониторинга и регистрации ошибок для отслеживания и анализа сбоев и ошибок в работе приложения.

Микросервисный подход:

    Тестирование отказоустойчивости каждого сервиса: Провести тестирование отказоустойчивости каждого сервиса отдельно, включая проверку его реакции на сбои и ошибки.
    Изоляция сбоев: Обеспечить изоляцию сбоев в отдельных сервисах для предотвращения распространения проблем на всю систему.

Вывод:

    В монолитном подходе надежность приложения оценивается как единое целое, что может упростить процесс мониторинга и анализа сбоев, но также увеличивает риск целостного отказа системы при сбое.
    В микросервисном подходе каждый сервис оценивается отдельно по его надежности, что позволяет более точно контролировать и управлять сбоями и проблемами, но также требует дополнительных усилий по организации и мониторингу.

\subsection{Юзабилити}

Монолитный подход:

    Тестирование пользовательского интерфейса: Провести тестирование пользовательского интерфейса приложения на соответствие стандартам юзабилити, включая проверку навигации, удобства использования элементов управления и визуального дизайна.
    Сбор обратной связи от пользователей: Провести опросы и интервью с пользователями для оценки их удовлетворенности и определения улучшений, которые можно внести в интерфейс приложения.

Микросервисный подход:

    Тестирование интерфейсов каждого сервиса: Провести тестирование интерфейсов каждого сервиса на удобство использования и соответствие потребностям конечных пользователей.
    Интеграционное тестирование: Провести тестирование взаимодействия между сервисами и оценить удобство использования системы в целом, включая процессы аутентификации, авторизации и передачи данных между сервисами.

Вывод:

    В монолитном подходе юзабилити оценивается как единое целое, что может упростить процесс тестирования и улучшения пользовательского опыта, но также может привести к более сложной навигации и интерфейсу.
    В микросервисном подходе каждый сервис оценивается отдельно по его юзабилити, что позволяет более точно адаптировать интерфейсы к потребностям пользователей, но также требует дополнительных усилий по интеграционному тестированию и обеспечению консистентности интерфейсов.

\subsection{Совместимость}

Монолитный подход:

    Тестирование на разных платформах: Проверить работоспособность приложения на различных операционных системах и браузерах для определения степени совместимости.
    Проверка совместимости с различными устройствами: Оценить работоспособность приложения на различных типах устройств (настольные компьютеры, планшеты, мобильные телефоны) и разрешениях экранов.

Микросервисный подход:

    Совместимость интерфейсов сервисов: Проверить совместимость интерфейсов и протоколов взаимодействия между сервисами для обеспечения их корректной работы в различных средах выполнения.
    Тестирование совместимости с внешними системами: Провести тестирование взаимодействия с внешними системами и сервисами для определения совместимости и обеспечения корректной работы приложения в различных окружениях.

Вывод:

    В монолитном подходе совместимость оценивается как единое целое, что может упростить процесс тестирования и обеспечения совместимости, но также может ограничить возможности адаптации к различным средам выполнения.
    В микросервисном подходе совместимость каждого сервиса оценивается отдельно, что позволяет более точно контролировать и управлять совместимостью и адаптацией к различным средам, но также требует дополнительных усилий по тестированию и обеспечению совместимости интерфейсов.

\subsection{Юридические вопросы}

Монолитный подход:

    Аудит правовой соответствия: Провести аудит кода и архитектуры приложения для выявления возможных юридических проблем, таких как нарушение авторских прав или конфиденциальности данных.
    Проверка лицензий и соглашений: Проверить соответствие использованных технологий и компонентов лицензионным соглашениям и требованиям конкретных проектов.

Микросервисный подход:

    Контроль доступа и авторизации: Обеспечить корректное управление доступом к сервисам и данным в соответствии с требованиями конфиденциальности и безопасности.
    Защита данных: Обеспечить адекватную защиту персональных данных и конфиденциальной информации в каждом из сервисов.

Вывод:

    В монолитном подходе юридические вопросы оцениваются как единое целое, что может упростить процесс аудита и контроля, но также может увеличить риск возникновения проблем из-за несоответствия правовым требованиям.
    В микросервисном подходе каждый сервис оценивается отдельно с точки зрения юридической безопасности и соответствия, что позволяет более точно контролировать и управлять рисками, но также требует дополнительных усилий по обеспечению соответствия и контроля.














    \subsection{Краткое содержание}
    Архитектурные особенности: Сравнение основных архитектурных принципов и подходов между монолитной и микросервисной архитектурами.
    Гибкость и масштабируемость: Оценка гибкости и возможностей масштабирования каждой из архитектур при изменении требований и нагрузки.
    Управление и развертывание: Сравнение процессов управления, развертывания и обновления приложений в контексте монолитной и микросервисной архитектур.
    Отказоустойчивость и производительность: Оценка уровня отказоустойчивости и производительности каждой архитектуры при различных условиях эксплуатации.
    Развитие и поддержка: Рассмотрение сложностей и преимуществ развития и поддержки приложений на основе монолитной и микросервисной архитектур.

    Сопоставление ключевых характеристик и результатов для монолитного и микросервисного подходов в контексте выбранных проектов:

    Архитектурные особенности:
        Монолитная архитектура: Единое приложение собрано в одном контейнере, что облегчает начальное развертывание и управление.
        Микросервисная архитектура: Приложение разделено на отдельные микросервисы, обеспечивая гибкость и независимость компонентов.

    Гибкость и масштабируемость:
        Монолитная архитектура: Ограниченная гибкость при масштабировании отдельных компонентов. Необходимость масштабировать всё приложение целиком.
        Микросервисная архитектура: Высокая гибкость в масштабировании и изменении отдельных компонентов без влияния на другие.

    Управление и развертывание:
        Монолитная архитектура: Проще в управлении и развертывании благодаря единому контейнеру, но требует перезапуска всего приложения при обновлении.
        Микросервисная архитектура: Более сложное управление из-за большего количества сервисов, но обновления могут быть более гибкими и безотказными.

    Отказоустойчивость и производительность:
        Монолитная архитектура: Отказ одной части может привести к отказу всего приложения. Масштабируемость ограничена.
        Микросервисная архитектура: Более высокая отказоустойчивость за счет изоляции отдельных сервисов. Возможность гибкого масштабирования отдельных компонентов для оптимизации производительности.

    Развитие и поддержка:
        Монолитная архитектура: Проще в разработке и поддержке для небольших и простых приложений с ограниченными требованиями.
        Микросервисная архитектура: Больше сложностей в разработке и поддержке из-за необходимости управления множеством сервисов, но более подходит для крупных и распределенных систем.

\subsection{Анализ производительности, масштабируемости, поддерживаемости и других аспектов каждой архитектуры}
    Монолитная архитектура:
    
        Производительность: Приложение может столкнуться с проблемами производительности из-за неэффективного масштабирования и обработки больших объемов данных в едином контейнере.
        Масштабируемость: Ограниченная масштабируемость из-за необходимости масштабировать всё приложение целиком, что может привести к избыточному использованию ресурсов.
        Поддерживаемость: Проще в поддержке из-за централизованной структуры, однако изменения в приложении могут потребовать пересборки и перезапуска всего приложения.
        Отказоустойчивость: Отказ одной части приложения может привести к полному отказу приложения, так как все компоненты работают в едином контексте.
    
    Микросервисная архитектура:
    
        Производительность: Более гибкая и эффективная производительность благодаря изоляции сервисов и возможности масштабирования только необходимых компонентов.
        Масштабируемость: Высокая масштабируемость, так как каждый сервис может быть масштабирован независимо от других, что позволяет оптимизировать использование ресурсов.
        Поддерживаемость: Более сложно в поддержке из-за необходимости управления множеством сервисов, но при правильном организации кода и процессов разработки это может быть облегчено.
        Отказоустойчивость: Более высокая отказоустойчивость за счет изоляции отдельных сервисов, что позволяет предотвратить полный отказ приложения из-за отказа одного компонента.

\subsection{Идентификация сильных и слабых сторон каждого подхода}
    Монолитная архитектура:

    Сильные стороны:
    
        Проще в разработке и начальном развертывании благодаря централизованной структуре.
        Меньше сложностей с управлением и развертыванием, так как всё приложение собрано в одном контейнере.
        Относительно более проста в поддержке для небольших и простых приложений с ограниченными требованиями.
    
    Слабые стороны:
    
        Ограниченная масштабируемость из-за необходимости масштабировать всё приложение целиком, что может привести к избыточному использованию ресурсов.
        Более сложно в управлении и поддержке при росте функциональности и увеличении объема данных.
        Больше риска полного отказа приложения из-за отказа одной из его частей.
    
    Микросервисная архитектура:
    
    Сильные стороны:
    
        Высокая гибкость и масштабируемость благодаря изоляции сервисов и возможности масштабирования только необходимых компонентов.
        Более высокая отказоустойчивость за счет изоляции отдельных сервисов, что позволяет предотвратить полный отказ приложения из-за отказа одного компонента.
        Более эффективное использование ресурсов при правильном масштабировании и настройке компонентов.
    
    Слабые стороны:
    
        Больше сложностей в управлении и развертывании из-за необходимости управления множеством сервисов и их зависимостей.
        Более высокие затраты на разработку и поддержку из-за необходимости управления множеством кодовых баз и интеграций между сервисами.
        Риск усложнения системы и возможности перехода к "монолитному" подходу при неправильном проектировании или неподходящем управлении.
        
% Обязательно добавляем это в конце каждой секции, чтобы 
% обеспечить переход на новую страницу
\clearpage % Анализ
    \anonsection{Заключение}

\subsection{Сводные выводы и обобщения результатов исследования}
    Выбор подхода в зависимости от требований проекта: Оба подхода имеют свои сильные и слабые стороны, поэтому выбор между ними должен зависеть от конкретных требований проекта, его масштаба, сложности и прочих факторов.

    Монолитная архитектура: Подходит для небольших и простых проектов с ограниченными требованиями, где важны быстрое развертывание и простота управления. Однако она имеет ограничения в масштабируемости и отказоустойчивости.

    Микросервисная архитектура: Подходит для крупных и сложных проектов, где важны гибкость, масштабируемость и отказоустойчивость. Однако она требует больше усилий в управлении, разработке и поддержке из-за большего числа сервисов и их зависимостей.

    Гибридные подходы: В некоторых случаях может быть целесообразно использовать гибридные подходы, комбинируя элементы монолитной и микросервисной архитектур для оптимального сочетания преимуществ и минимизации недостатков.

    Непрерывное развитие и адаптация: Независимо от выбранного подхода, важно осуществлять непрерывное развитие и адаптацию системы под изменяющиеся требования и условия, что позволит обеспечить ее эффективность и успешное функционирование в долгосрочной перспективе.



\subsection{Рекомендации по выбору архитектуры для подобных проектов в зависимости от контекста}

    Размер и сложность проекта: Для небольших и простых проектов с ограниченными требованиями к ресурсам и функциональности может быть предпочтительнее использовать монолитную архитектуру.

    Гибкость и масштабируемость: Если проект предполагает рост функциональности, обработку больших объемов данных или необходимость масштабируемости, микросервисная архитектура может быть более подходящим выбором.

    Отказоустойчивость и надежность: Для проектов, где требуется высокая отказоустойчивость и возможность избежать полного отказа системы при отказе отдельных компонентов, микросервисная архитектура предпочтительнее.

    Скорость развертывания и поддержка: Если важно быстрое развертывание и управление системой с минимальными усилиями, то монолитная архитектура может быть более подходящим вариантом.

    Бюджет и ресурсы: Учитывайте бюджет проекта и наличие необходимых ресурсов (времени, денег, персонала) для реализации и поддержки выбранной архитектуры.

    Гибридные подходы: Использование гибридных подходов, комбинируя элементы монолитной и микросервисной архитектур, может быть оптимальным решением в случае комплексных требований и условий проекта.

    Непрерывное развитие: Независимо от выбранной архитектуры, важно осуществлять непрерывное развитие и адаптацию системы под изменяющиеся требования и условия, чтобы обеспечить ее эффективность и успешное функционирование в долгосрочной перспективе.



\subsection{Перспективы дальнейших исследований в данной области}

    Оптимизация производительности: Дальнейшие исследования могут быть направлены на оптимизацию производительности и эффективности обеих архитектур, включая улучшение алгоритмов, инфраструктуры и масштабируемости.

    Исследование гибридных подходов: Исследования могут быть проведены по анализу и разработке гибридных подходов, комбинирующих элементы монолитной и микросервисной архитектур для оптимального сочетания преимуществ и минимизации недостатков.

    Автоматизация управления и развертывания: Исследования могут быть направлены на разработку инструментов и методов автоматизации управления и развертывания приложений в обеих архитектурах, что позволит снизить затраты на поддержку и повысить эффективность процессов разработки.

    Исследование влияния архитектуры на безопасность: Дальнейшие исследования могут оценить влияние выбора архитектуры на безопасность приложений и возможности предотвращения атак и утечек данных.

    Исследование эволюции архитектурных подходов: Исследования могут быть направлены на анализ эволюции монолитных и микросервисных архитектур в контексте изменяющихся требований и тенденций в разработке программного обеспечения.

    Исследование влияния архитектуры на процессы разработки: Исследования могут оценить влияние выбора архитектуры на процессы разработки, включая скорость развертывания, стоимость разработки и поддержки, а также качество и надежность конечного продукта.

Дальнейшие исследования в этих направлениях помогут углубить понимание преимуществ и недостатков различных архитектурных подходов и способствуют развитию более эффективных методов разработки программного обеспечения.








\section{TMP}



Напиши "Обзор монолитного и микросервисного подходов к разработке веб-приложений."
ChatGPT

Обзор монолитного и микросервисного подходов к разработке веб-приложений:

Монолитный подход представляет собой традиционную архитектурную модель, в которой вся функциональность приложения размещается в едином исполняемом контейнере. В такой архитектуре весь код, отвечающий за различные компоненты приложения, объединен в одном монолитном приложении. Обычно монолитные приложения разрабатываются на одном языке программирования и обращаются к одной базе данных.

Микросервисный подход, напротив, предполагает разделение приложения на небольшие, независимые сервисы, каждый из которых отвечает за определенную функциональность. Каждый сервис может быть разработан на отдельном языке программирования, иметь свою базу данных и взаимодействовать с другими сервисами посредством API. Такая архитектура обеспечивает гибкость, масштабируемость и упрощает поддержку приложения.

Монолитные приложения обычно более просты в развертывании и управлении, но могут столкнуться с проблемами масштабирования и поддержки при увеличении объема функциональности. Микросервисные приложения позволяют более гибко масштабировать отдельные компоненты, но требуют дополнительных усилий по управлению распределенной системой и обеспечению целостности данных.

Выбор между монолитным и микросервисным подходами зависит от конкретных требований проекта, его масштаба, потребностей в гибкости и масштабируемости, а также от доступных ресурсов и экспертизы команды разработчиков.




Напиши "Теоретическая часть:"
ChatGPT

Теоретическая часть:

В данном разделе проводится детальный анализ основных концепций и технологий, связанных с монолитным и микросервисным подходами к разработке веб-приложений. Рассматриваются ключевые аспекты архитектуры, принципы организации кода, взаимодействие компонентов, а также инструменты и технологии, используемые для реализации каждого из этих подходов. Этот раздел поможет понять основные принципы работы монолитных и микросервисных приложений, их отличия и особенности, а также их применимость в различных сценариях разработки и эксплуатации.
Напиши "Обзор основных концепций и технологий, используемых в монолитном и микросервисном программировании (например, Java, VueJS, PostgreSQL, RESTful API и т.д.)."
ChatGPT

Обзор основных концепций и технологий, используемых в монолитном и микросервисном программировании:

    Монолитное программирование:
        Языки программирования: часто используются Java, C\#, Python, PHP и другие.
        Фронтенд: для разработки пользовательского интерфейса часто применяются фреймворки и библиотеки, такие как Vue.js, React, Angular.
        Базы данных: для хранения данных могут использоваться реляционные СУБД (например, PostgreSQL, MySQL) или NoSQL решения (например, MongoDB, Cassandra).
        RESTful API: для взаимодействия между клиентом и сервером часто применяются RESTful API, которые обеспечивают стандартизированный интерфейс для обмена данными.

    Микросервисное программирование:
        Языки программирования: микросервисы могут быть написаны на различных языках программирования в зависимости от их целей и требований. Это может включать в себя Java, Python, Go, Node.js и другие.
        Фронтенд: микросервисная архитектура не определяет конкретные технологии для фронтенда, поскольку фронтенд может быть разделен на отдельные сервисы или реализован с использованием серверного рендеринга.
        Базы данных: каждый микросервис может иметь свою собственную базу данных, что позволяет выбирать подходящее хранилище данных для каждого сервиса.
        RESTful API или gRPC: для взаимодействия между микросервисами часто используются RESTful API или более эффективные альтернативы, такие как gRPC.

Обе архитектуры имеют свои уникальные особенности и требования к технологическому стеку, которые определяются целями проекта, его масштабом и требованиями к производительности и масштабируемости.
Напиши "Подробное описание архитектурных принципов каждого подхода."
ChatGPT

Подробное описание архитектурных принципов каждого подхода:

Монолитная архитектура:

    Единственный исполняемый контейнер: Вся функциональность приложения объединена в едином исполняемом контейнере. Все компоненты, такие как интерфейс пользователя, бизнес-логика и доступ к данным, находятся внутри этого контейнера.
    Монолитный код: Приложение обычно разрабатывается на одном языке программирования. Весь код, относящийся к разным компонентам приложения, находится внутри этого монолитного кодового базы.
    Централизованная база данных: В монолитной архитектуре часто используется единая централизованная база данных, к которой обращаются все компоненты приложения.

Микросервисная архитектура:

    Разделение на независимые сервисы: Приложение разбивается на небольшие, независимые сервисы, каждый из которых отвечает за определенную функциональность или предметную область.
    Распределенная архитектура: Каждый микросервис может иметь свою собственную базу данных и свою собственную кодовую базу. Сервисы взаимодействуют друг с другом через сетевые вызовы, обычно посредством API.
    Гибкость в выборе технологий: Различные микросервисы могут быть реализованы на разных языках программирования и использовать разные технологии в зависимости от их конкретных потребностей и целей.
    Отказоустойчивость: Отказ одного сервиса не приводит к полной недоступности всего приложения. Система может продолжать работать, даже если один из сервисов временно недоступен.

Эти принципы определяют основные архитектурные концепции каждого подхода и оказывают существенное влияние на структуру, масштабируемость, гибкость и управляемость приложения.
Напиши "Анализ сценариев использования и возможных решений задач для каждой архитектуры."
ChatGPT

Анализ сценариев использования и возможных решений задач для каждой архитектуры:

Монолитная архитектура:

    Простые приложения с низкими требованиями к масштабируемости: Монолитная архитектура подходит для разработки небольших приложений с ограниченным числом функций и низкой нагрузкой.
    Прототипирование и быстрое развертывание: В случаях, когда требуется быстро развернуть прототип или MVP, монолитная архитектура может быть быстрым и простым решением.
    Ограниченные ресурсы команды разработки: Если у команды разработки ограничены ресурсы и нет опыта работы с распределенными системами, монолитная архитектура может быть более доступным и простым вариантом.

Микросервисная архитектура:

    Большие и сложные приложения: Микросервисная архитектура подходит для разработки больших и сложных приложений, где разные части приложения могут развиваться и масштабироваться независимо друг от друга.
    Высокая нагрузка и масштабируемость: Если требуется высокая масштабируемость и возможность быстрого масштабирования отдельных компонентов приложения, микросервисная архитектура предоставляет более гибкое решение.
    Распределенные команды разработки: Если у команды разработки есть опыт работы с распределенными системами и каждая часть приложения может быть разработана и поддерживаться отдельной командой, микросервисная архитектура может быть предпочтительным выбором.

Анализ сценариев использования позволяет определить, какая из архитектурных моделей лучше подходит для конкретного проекта, учитывая его требования, масштаб и доступные ресурсы команды разработки.
Напиши "Описание архитектуры выбранных проектов (медицинские перевозки и каталогизация данных)."
ChatGPT




Описание архитектуры выбранных проектов:

    Проект о медицинских перевозках (монолитная структура):
        Фронтенд: Реализован на Vue.js для создания интерфейса пользователя с использованием компонентного подхода.
        Бэкенд: Написан на Java с использованием фреймворка Spring Boot для реализации бизнес-логики и обработки запросов от клиента.
        База данных: Используется PostgreSQL для хранения информации о медицинских перевозках, пациентах, медицинских учреждениях и т.д.
        Взаимодействие: Фронтенд общается с бэкендом посредством RESTful API, который обеспечивает передачу данных между клиентом и сервером.
        Особенности: Вся функциональность приложения объединена в едином исполняемом контейнере, что облегчает развертывание и управление приложением. Однако, при увеличении объема функциональности могут возникнуть проблемы с масштабируемостью.

    Проект о каталогизации и структуризации данных (микросервисная архитектура):
        Регистрация сервиса: Каждый аспект каталогизации данных, такой как управление категориями, атрибутами и сущностями, представлен в виде отдельного микросервиса.
        Независимая база данных: Каждый микросервис имеет свою собственную базу данных (например, MongoDB, MySQL), что позволяет избежать проблем, связанных с централизованным хранилищем данных.
        Коммуникация через API: Сервисы взаимодействуют между собой посредством API, обеспечивая передачу данных и вызов методов между различными компонентами приложения.
        Гибкость и масштабируемость: Каждый сервис может масштабироваться независимо от других, что обеспечивает гибкость в управлении ресурсами и обеспечивает высокую отказоустойчивость и производительность приложения.

Эти описания архитектуры демонстрируют особенности каждого проекта и выбранной для него архитектурной модели, а также позволяют понять, какая архитектура лучше соответствует требованиям и целям каждого проекта.
















Напиши параметры, по которым можно сравнить два приложения, чтобы понять, какой подход имеет преимущество. Упомяни как функциональные, так и нефункциональные требования
ChatGPT


Для сравнения двух приложений и определения преимущества того или иного подхода (монолитного или микросервисного) можно учитывать следующие параметры, включая как функциональные, так и нефункциональные требования:

Функциональные требования:

    Функциональность приложения: Сравнение функциональных возможностей и способности приложений решать задачи и удовлетворять потребности пользователей.
    Гибкость и масштабируемость: Оценка возможности приложений адаптироваться к изменяющимся требованиям и масштабироваться для работы с различными объемами данных и нагрузками.
    Изоляция компонентов: Учитывание степени изоляции компонентов приложения и их независимости друг от друга для обеспечения гибкости разработки и поддержки.
    Развитие и поддержка: Оценка удобства разработки новых функций и поддержки существующих в рамках выбранной архитектуры.
    Интеграция с внешними системами: Учет возможности интеграции приложений с другими системами и сервисами для обеспечения расширенных функциональных возможностей.

Нефункциональные требования:

    Производительность: Сравнение производительности приложений в условиях различной нагрузки и объема данных.
    Отказоустойчивость и надежность: Оценка способности приложений обрабатывать ошибки и отказы без серьезных последствий для работы системы в целом.
    Безопасность: Анализ мер безопасности и возможности обеспечения защиты данных и системы от внешних угроз.
    Скорость развертывания и обновления: Оценка скорости и удобства развертывания новых версий приложений и обновлений.
    Затраты на разработку и поддержку: Учет затрат на разработку, развертывание и поддержку приложений в течение их жизненного цикла.

Учитывая эти параметры, можно определить, какой подход (монолитный или микросервисный) лучше соответствует требованиям и целям конкретного проекта.
Нефункциональное требование относится к системе целиком. Оно не описывает, как пользователь будет взаимодействовать с системой или интерфейсом. Нефункциональные требования можно разделить на две категории: Атрибуты качества: Это характеристики системы, определяющие ее общее качество. Примеры атрибутов качества включают безопасность, производительность и удобство использования. Ограничения: Это ограничения, наложенные на систему. Примеры ограничений включают время, ресурсы и среду. Примеры таких требований: Безопасность: Система должна быть защищена от несанкционированного доступа. Производительность: Система должна быть способна обслуживать необходимое количество пользователей без снижения производительности. Масштабируемость: Система должна иметь возможность увеличивать или уменьшать масштаб по мере необходимости. Доступность: Система должна быть доступна, когда это необходимо. Техническое обслуживание: Система должна быть проста в обслуживании и обновлении. Переносимость: Система должна работать на разных платформах с минимальными изменениями. Надежность: Система должна быть надежной и соответствовать требованиям пользователя. Юзабилити: Система должна быть простой в использовании и понятной. Совместимость: Система должна быть совместима с другими системами. Юридические вопросы: Система должна соответствовать всем применимым законам и правилам. Распиши кратко по каждому из нейункциональных требований, как можно проверить выполнимость этих требований для каждого из проектов для объективной оценки и сравнения монолитного и микросервисного подходов
ChatGPT

Для объективной оценки и сравнения монолитного и микросервисного подходов по нефункциональным требованиям можно использовать следующие критерии:

    Безопасность:
        Монолитный подход: Аудит безопасности приложения в целом, включая проверку уязвимостей и обеспечение соответствия стандартам безопасности.
        Микросервисный подход: Анализ безопасности каждого сервиса отдельно, а также проверка наличия механизмов защиты данных при передаче между сервисами.

    Производительность:
        Монолитный подход: Оценка производительности единого приложения под нагрузкой, включая время отклика и скорость выполнения различных операций.
        Микросервисный подход: Оценка производительности каждого сервиса отдельно и их взаимодействия, а также оценка пропускной способности сети между сервисами.

    Масштабируемость:
        Монолитный подход: Оценка возможности масштабирования всего приложения целиком, включая возможность горизонтального и вертикального масштабирования.
        Микросервисный подход: Оценка гибкости и простоты масштабирования отдельных сервисов в зависимости от нагрузки на них, а также возможность автоматического масштабирования при необходимости.

    Доступность:
        Монолитный подход: Анализ уязвимых мест и резервных систем для обеспечения доступности приложения в случае отказов или сбоев.
        Микросервисный подход: Обеспечение высокой доступности путем дублирования и отказоустойчивости каждого сервиса, а также балансировки нагрузки между ними.

    Техническое обслуживание:
        Монолитный подход: Оценка сложности и затрат на обновление и поддержку всего приложения целиком, включая время, ресурсы и возможность автоматизации процессов.
        Микросервисный подход: Оценка сложности и затрат на обновление и поддержку каждого сервиса отдельно, а также возможность автоматизации и оркестрации процессов управления сервисами.

    Переносимость:
        Монолитный подход: Оценка возможности переноса приложения на различные платформы без необходимости крупных изменений в коде или архитектуре.
        Микросервисный подход: Оценка возможности переноса и совместимости каждого сервиса с различными платформами и средами выполнения.

    Надежность:
        Монолитный подход: Оценка стабильности и надежности работы всего приложения в целом, включая отказоустойчивость и восстановление после сбоев.
        Микросервисный подход: Оценка надежности каждого сервиса и их взаимодействия, а также механизмов обработки ошибок и восстановления после отказов.

    Юзабилити:
        Монолитный подход: Оценка удобства использования интерфейса и функциональности приложения для пользователей.
        Микросервисный подход: Анализ удобства использования и взаимодействия различных сервисов в рамках приложения для разработчиков и администраторов системы.

    Совместимость:
        Монолитный подход: Оценка совместимости приложения с другими системами и сервисами, с которыми оно взаимодействует.
        Микросервисный подход: Оценка совместимости каждого сервиса с другими системами и сервисами, а также возможность интеграции и обмена данными между ними.

    Юридические вопросы:
        Монолитный подход: Проверка соответствия всей архитектуры приложения применимым законам и правилам.
        Микросервисный подход: Анализ соответствия каждого сервиса применимым законам и правилам, а также возможность управления доступом к данным и сервисам с точки зрения правовых норм.

Анализ этих критериев для каждого из проектов позволит объективно оценить преимущества и недостатки монолитного и микросервисного подходов и выбрать наиболее подходящий вариант для конкретного случая.
Напиши подробнее про критерий "Безопасность". Упомяни, как конкретно можно объективно оценить данный критерий в каждом из проектов, и какой вывод можно сделать
ChatGPT

Для критерия "Безопасность" в каждом из проектов можно провести следующие оценки:

Монолитный подход:

    Аудит безопасности приложения: Провести комплексный аудит безопасности приложения, включая анализ уязвимостей кода, проверку на соответствие стандартам безопасности и наличие механизмов защиты.
    Обеспечение безопасности данных: Оценить методы шифрования и защиты данных при хранении и передаче, а также наличие механизмов контроля доступа к конфиденциальной информации.

Микросервисный подход:

    Анализ безопасности каждого сервиса: Проверить каждый сервис на наличие уязвимостей и возможные точки входа для атак, а также оценить механизмы обеспечения безопасности внутри сервиса.
    Защита взаимодействия между сервисами: Обеспечить безопасность передачи данных между сервисами, включая использование протоколов шифрования и аутентификации.

Вывод:

    В монолитном подходе безопасность приложения оценивается как единое целое, что может упростить процесс аудита и обеспечения безопасности, но также повышает риск распространения уязвимостей на всю систему.
    В микросервисном подходе необходимо обеспечить безопасность каждого сервиса отдельно, что требует дополнительных усилий, но также обеспечивает более гибкий и гранулярный контроль над безопасностью в целом.

Напиши подробнее про критерий "Производительность". Упомяни, как конкретно можно объективно оценить данный критерий в каждом из проектов, и какой вывод можно сделать
ChatGPT

Для критерия "Производительность" в каждом из проектов можно провести следующие оценки:

Монолитный подход:

    Тестирование времени отклика: Измерить время отклика приложения на запросы пользователей при различных нагрузках для определения скорости работы системы.
    Анализ использования ресурсов: Оценить использование процессора, памяти и сети приложением во время работы для выявления узких мест и возможных улучшений.
    Скорость обработки запросов: Измерить время выполнения различных операций приложения, таких как загрузка страниц, выполнение запросов к базе данных и обработка данных.

Микросервисный подход:

    Тестирование производительности каждого сервиса: Измерить время отклика и производительность каждого сервиса отдельно при различных нагрузках для определения уровня его эффективности.
    Анализ времени передачи данных между сервисами: Оценить время передачи данных между сервисами и задержки в коммуникации для определения эффективности взаимодействия.
    Скорость обработки запросов и сборка данных: Измерить время выполнения запросов, необходимых для выполнения конкретной операции, и время, необходимое для сборки данных из разных сервисов.

Вывод:

    В монолитном подходе производительность приложения оценивается как единое целое, что может упростить процесс оптимизации, но также ограничивает возможности масштабирования и гибкости.
    В микросервисном подходе производительность каждого сервиса оценивается отдельно, что позволяет более гибко управлять ресурсами и масштабировать отдельные компоненты, но требует дополнительных усилий по управлению взаимодействием и коммуникацией между сервисами.

Напиши подробнее про критерий "Масштабируемость". Упомяни, как конкретно можно объективно оценить данный критерий в каждом из проектов, и какой вывод можно сделать
ChatGPT

Для критерия "Масштабируемость" в каждом из проектов можно провести следующие оценки:

Монолитный подход:

    Горизонтальное масштабирование: Оценить возможность добавления дополнительных серверов для распределения нагрузки и увеличения пропускной способности приложения.
    Вертикальное масштабирование: Измерить возможность увеличения ресурсов на сервере (процессор, память) для обработки большего количества запросов и данных.

Микросервисный подход:

    Гибкость добавления новых сервисов: Оценить удобство и скорость добавления новых сервисов для расширения функциональности и увеличения масштабируемости.
    Гибкость масштабирования отдельных сервисов: Измерить возможность масштабирования отдельных сервисов по мере увеличения нагрузки на них.

Вывод:

    В монолитном подходе масштабируемость ограничена возможностями горизонтального и вертикального масштабирования всего приложения целиком, что может привести к ограничениям в управлении ресурсами и недостаточной гибкости.
    В микросервисном подходе масштабируемость повышается за счет возможности добавления и масштабирования отдельных сервисов по мере необходимости, что обеспечивает более гибкое управление ресурсами и адаптацию к изменяющимся требованиям.

Напиши подробнее про критерий "Доступность". Упомяни, как конкретно можно объективно оценить данный критерий в каждом из проектов, и какой вывод можно сделать
ChatGPT

Для оценки критерия "Доступность" в каждом из проектов можно использовать следующие методы:

Монолитный подход:

    Тестирование отказоустойчивости: Провести тестирование на отказы и сбои приложения, включая проверку реакции системы на непредвиденные обстоятельства, такие как отключение серверов или сетевых узлов.
    Мониторинг доступности: Оценить доступность приложения в реальном времени с помощью мониторинговых инструментов, таких как системы мониторинга производительности и управления событиями.

Микросервисный подход:

    Тестирование отказоустойчивости каждого сервиса: Провести тестирование отказоустойчивости каждого сервиса отдельно, включая проверку реакции на сбои и ошибки.
    Балансировка нагрузки: Использовать механизмы балансировки нагрузки для обеспечения равномерного распределения нагрузки между сервисами и предотвращения перегрузок.

Вывод:

    В монолитном подходе доступность приложения оценивается как единое целое, что может упростить процесс управления и контроля, но также увеличивает риск целостного отказа системы при сбое.
    В микросервисном подходе каждый сервис оценивается отдельно по его доступности, что позволяет более точно контролировать и управлять отказами и сбоями, но также требует дополнительных усилий по управлению и мониторингу.

Напиши подробнее про критерий "Техническое обслуживание". Упомяни, как конкретно можно объективно оценить данный критерий в каждом из проектов, и какой вывод можно сделать
ChatGPT

Для оценки критерия "Техническое обслуживание" в каждом из проектов можно использовать следующие методы:

Монолитный подход:

    Сложность обновлений и патчей: Оценить сложность процесса обновления и установки патчей для всего приложения целиком, включая возможность автоматизации этих процессов.
    Общее время простоя: Измерить время простоя приложения во время обновлений и технического обслуживания для определения общего влияния на производственные процессы.

Микросервисный подход:

    Управление версиями сервисов: Оценить возможность управления версиями сервисов и их зависимостями, включая автоматизацию процессов обновления и деплоймента.
    Мониторинг здоровья сервисов: Использовать мониторинговые инструменты для отслеживания здоровья каждого сервиса и оперативного реагирования на сбои и проблемы.

Вывод:

    В монолитном подходе техническое обслуживание оценивается как единое целое, что может упростить процесс управления и контроля, но также увеличивает риск простоев системы при обновлениях.
    В микросервисном подходе управление и мониторинг сервисов может быть более гибким и эффективным, но также требует дополнительных усилий по организации и автоматизации процессов обслуживания.

Напиши подробнее про критерий "Переносимость". Упомяни, как конкретно можно объективно оценить данный критерий в каждом из проектов, и какой вывод можно сделать
ChatGPT

Для оценки критерия "Переносимость" в каждом из проектов можно использовать следующие методы:

Монолитный подход:

    Тестирование на разных платформах: Проверить работоспособность приложения на различных операционных системах и аппаратных платформах для определения степени переносимости.
    Использование кросс-платформенных технологий: Оценить использование кросс-платформенных технологий и инструментов разработки для обеспечения универсальности приложения.

Микросервисный подход:

    Независимость от платформы: Проверить, насколько каждый сервис является независимым от платформы, и как легко его можно перенести на другие среды выполнения.
    Стандартизация интерфейсов: Использовать стандартизированные протоколы и интерфейсы для обеспечения совместимости и переносимости сервисов.

Вывод:

    В монолитном подходе переносимость приложения оценивается как единое целое, что может упростить процесс тестирования и обеспечения совместимости, но также может ограничить возможности адаптации к различным средам выполнения.
    В микросервисном подходе каждый сервис оценивается отдельно по его переносимости, что позволяет более гибко управлять совместимостью и адаптацией к различным платформам, но также требует дополнительных усилий по обеспечению независимости и стандартизации.

Напиши подробнее про критерий "Надежность". Упомяни, как конкретно можно объективно оценить данный критерий в каждом из проектов, и какой вывод можно сделать
ChatGPT

Для оценки критерия "Надежность" в каждом из проектов можно использовать следующие методы:

Монолитный подход:

    Тестирование отказоустойчивости: Провести тестирование на отказы и сбои приложения, включая проверку реакции системы на непредвиденные обстоятельства, такие как отключение серверов или сетевых узлов.
    Мониторинг и регистрация ошибок: Установить систему мониторинга и регистрации ошибок для отслеживания и анализа сбоев и ошибок в работе приложения.

Микросервисный подход:

    Тестирование отказоустойчивости каждого сервиса: Провести тестирование отказоустойчивости каждого сервиса отдельно, включая проверку его реакции на сбои и ошибки.
    Изоляция сбоев: Обеспечить изоляцию сбоев в отдельных сервисах для предотвращения распространения проблем на всю систему.

Вывод:

    В монолитном подходе надежность приложения оценивается как единое целое, что может упростить процесс мониторинга и анализа сбоев, но также увеличивает риск целостного отказа системы при сбое.
    В микросервисном подходе каждый сервис оценивается отдельно по его надежности, что позволяет более точно контролировать и управлять сбоями и проблемами, но также требует дополнительных усилий по организации и мониторингу.

Напиши подробнее про критерий "Юзабилити". Упомяни, как конкретно можно объективно оценить данный критерий в каждом из проектов, и какой вывод можно сделать
ChatGPT

Для оценки критерия "Юзабилити" в каждом из проектов можно использовать следующие методы:

Монолитный подход:

    Тестирование пользовательского интерфейса: Провести тестирование пользовательского интерфейса приложения на соответствие стандартам юзабилити, включая проверку навигации, удобства использования элементов управления и визуального дизайна.
    Сбор обратной связи от пользователей: Провести опросы и интервью с пользователями для оценки их удовлетворенности и определения улучшений, которые можно внести в интерфейс приложения.

Микросервисный подход:

    Тестирование интерфейсов каждого сервиса: Провести тестирование интерфейсов каждого сервиса на удобство использования и соответствие потребностям конечных пользователей.
    Интеграционное тестирование: Провести тестирование взаимодействия между сервисами и оценить удобство использования системы в целом, включая процессы аутентификации, авторизации и передачи данных между сервисами.

Вывод:

    В монолитном подходе юзабилити оценивается как единое целое, что может упростить процесс тестирования и улучшения пользовательского опыта, но также может привести к более сложной навигации и интерфейсу.
    В микросервисном подходе каждый сервис оценивается отдельно по его юзабилити, что позволяет более точно адаптировать интерфейсы к потребностям пользователей, но также требует дополнительных усилий по интеграционному тестированию и обеспечению консистентности интерфейсов.

Напиши подробнее про критерий "Совместимость". Упомяни, как конкретно можно объективно оценить данный критерий в каждом из проектов, и какой вывод можно сделать
ChatGPT

Для оценки критерия "Совместимость" в каждом из проектов можно использовать следующие методы:

Монолитный подход:

    Тестирование на разных платформах: Проверить работоспособность приложения на различных операционных системах и браузерах для определения степени совместимости.
    Проверка совместимости с различными устройствами: Оценить работоспособность приложения на различных типах устройств (настольные компьютеры, планшеты, мобильные телефоны) и разрешениях экранов.

Микросервисный подход:

    Совместимость интерфейсов сервисов: Проверить совместимость интерфейсов и протоколов взаимодействия между сервисами для обеспечения их корректной работы в различных средах выполнения.
    Тестирование совместимости с внешними системами: Провести тестирование взаимодействия с внешними системами и сервисами для определения совместимости и обеспечения корректной работы приложения в различных окружениях.

Вывод:

    В монолитном подходе совместимость оценивается как единое целое, что может упростить процесс тестирования и обеспечения совместимости, но также может ограничить возможности адаптации к различным средам выполнения.
    В микросервисном подходе совместимость каждого сервиса оценивается отдельно, что позволяет более точно контролировать и управлять совместимостью и адаптацией к различным средам, но также требует дополнительных усилий по тестированию и обеспечению совместимости интерфейсов.

Напиши подробнее про критерий "Юридические вопросы". Упомяни, как конкретно можно объективно оценить данный критерий в каждом из проектов, и какой вывод можно сделать
ChatGPT

Для оценки критерия "Юридические вопросы" в каждом из проектов можно использовать следующие методы:

Монолитный подход:

    Аудит правовой соответствия: Провести аудит кода и архитектуры приложения для выявления возможных юридических проблем, таких как нарушение авторских прав или конфиденциальности данных.
    Проверка лицензий и соглашений: Проверить соответствие использованных технологий и компонентов лицензионным соглашениям и требованиям конкретных проектов.

Микросервисный подход:

    Контроль доступа и авторизации: Обеспечить корректное управление доступом к сервисам и данным в соответствии с требованиями конфиденциальности и безопасности.
    Защита данных: Обеспечить адекватную защиту персональных данных и конфиденциальной информации в каждом из сервисов.

Вывод:

    В монолитном подходе юридические вопросы оцениваются как единое целое, что может упростить процесс аудита и контроля, но также может увеличить риск возникновения проблем из-за несоответствия правовым требованиям.
    В микросервисном подходе каждый сервис оценивается отдельно с точки зрения юридической безопасности и соответствия, что позволяет более точно контролировать и управлять рисками, но также требует дополнительных усилий по обеспечению соответствия и контроля.



    

% Обязательно добавляем это в конце каждой секции, чтобы 
% обеспечить переход на новую страницу
\clearpage % Заключение
    
    \newpage
\renewcommand{\section}[2]{\anonsection{Библиографический список}}

%%% Подключение списка литературы по госту 2008 года
\bibliographystyle{ugost2008l}

%Подключаем сам файл с литературой (разрешение .bib)
%%% В ФАЙЛЕ ДЛЯ РУССКИХ ИСТОЧНИКОВ ДОБАВЛЯЕМ language={russian}
\bibliography{pattern/references}

%\label{sec:bibliography}
%\printbibliography

% Обязательно добавляем это в конце каждой секции, чтобы 
% обеспечить переход на новую страницу
\clearpage % Подключение библиографии
    
    %\input{inc/a-app} % Подключение приложения

\end{document}
