\renewcommand{\section}[2]{\anonsection{Библиографический список}}
%%% Подключение списка литературы по госту 2008 года
\bibliographystyle{ugost2008l}
% Подключаем сам файл с литературой (разрешение .bib)

%%% В ФАЙЛЕ ДЛЯ РУССКИХ ИСТОЧНИКОВ ДОБАВЛЯЕМ language={russian}
\bibliography{references}

\label{sec:bibliography}

https://madnight.github.io/githut/\#/stars/2024/1

https://siteefy.com/how-many-websites-are-there/

%https://blog.fora-soft.ru/post/chto-takoe-nefunkcionalnye-trebovaniya

%https://bestprogrammer.ru/izuchenie/funktsionalnye-i-nefunktsionalnye-trebovaniya-polnoe-rukovodstvo

%https://babok-school.ru/blog/what-is-non-functional-requirement-and-how-to-specify-it/



% Обязательно добавляем это в конце каждой секции, чтобы 
% обеспечить переход на новую страницу
\clearpage