\renewcommand{\section}[2]{\anonsection{Библиографический список}}
%%% Подключение списка литературы по госту 2008 года
\bibliographystyle{ugost2008l}
% Подключаем сам файл с литературой (разрешение .bib)

%%% В ФАЙЛЕ ДЛЯ РУССКИХ ИСТОЧНИКОВ ДОБАВЛЯЕМ language={russian}
\bibliography{references}

\label{sec:bibliography}

https://visuresolutions.com/ru/%D1%80%D1%83%D0%BA%D0%BE%D0%B2%D0%BE%D0%B4%D1%81%D1%82%D0%B2%D0%BE-%D0%BF%D0%BE-%D0%BE%D1%82%D1%81%D0%BB%D0%B5%D0%B6%D0%B8%D0%B2%D0%B0%D0%B5%D0%BC%D0%BE%D1%81%D1%82%D0%B8-%D1%83%D0%BF%D1%80%D0%B0%D0%B2%D0%BB%D0%B5%D0%BD%D0%B8%D1%8F-%D1%82%D1%80%D0%B5%D0%B1%D0%BE%D0%B2%D0%B0%D0%BD%D0%B8%D1%8F%D0%BC%D0%B8/%D0%BD%D0%B5%D1%84%D1%83%D0%BD%D0%BA%D1%86%D0%B8%D0%BE%D0%BD%D0%B0%D0%BB%D1%8C%D0%BD%D1%8B%D0%B5-%D1%82%D1%80%D0%B5%D0%B1%D0%BE%D0%B2%D0%B0%D0%BD%D0%B8%D1%8F/

https://blog.fora-soft.ru/post/chto-takoe-nefunkcionalnye-trebovaniya

https://bestprogrammer.ru/izuchenie/funktsionalnye-i-nefunktsionalnye-trebovaniya-polnoe-rukovodstvo

https://babok-school.ru/blog/what-is-non-functional-requirement-and-how-to-specify-it/



% Обязательно добавляем это в конце каждой секции, чтобы 
% обеспечить переход на новую страницу
\clearpage