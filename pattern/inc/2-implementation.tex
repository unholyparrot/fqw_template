\section{Реализация}

\subsection{Монолит}
    Проект о медицинских перевозках (монолитная структура):

    Анализ требований: Начало процесса с обсуждения и выявления требований к приложению, определения его функциональности и основных характеристик.
    Проектирование архитектуры: Разработка общей структуры приложения, включая разделение на слои (фронтенд, бэкенд), определение моделей данных и выбор архитектурных паттернов.
    Выбор технологий: На основе требований и анализа архитектуры выбираются технологии. Для фронтенда выбран Vue.js, для бэкенда - Java с использованием Spring Boot, а в качестве базы данных - PostgreSQL.
    Разработка: Каждый компонент приложения разрабатывается отдельно, начиная с фронтенда и заканчивая бэкендом. Разработка ведется в циклах итераций, с регулярным тестированием и интеграцией новых функций.
    Тестирование и отладка: После завершения разработки проводится тестирование каждого компонента отдельно, а затем интеграционное тестирование всего приложения для обнаружения ошибок и недочетов.
    Развертывание и поддержка: После успешного тестирования приложение разворачивается на сервере, где поддерживается и обновляется в соответствии с требованиями и обратной связью от пользователей.

    \begin{lstlisting}
    css
    
    monolith_project/
    │
    ├── frontend/
    │   ├── src/
    │   │   ├── components/
    │   │   │   ├── Header.vue
    │   │   │   ├── Sidebar.vue
    │   │   │   └── ...
    │   │   └── ...
    │   └── ...
    │
    ├── backend/
    │   ├── src/
    │   │   ├── controllers/
    │   │   │   ├── UserController.java
    │   │   │   ├── AppointmentController.java
    │   │   │   └── ...
    │   │   ├── models/
    │   │   │   ├── User.java
    │   │   │   ├── Appointment.java
    │   │   │   └── ...
    │   │   └── ...
    │   └── ...
    │
    └── database/
        ├── migrations/
        │   ├── 202205241200_create_users_table.sql
        │   ├── 202205241201_create_appointments_table.sql
        │   └── ...
        └── ...
    \end{lstlisting}
    
    Скрипт для запуска:
    \begin{lstlisting}
    export TEST_ENVIRONMENT=0
    export MTS_SERVER_PORT=8082
    
    export SUDO_ENVIRONMENT=1
    spring_args="$spring_args sudo"
    
    if [ "$(id -u)" -ne 0 ];
    then
        # not sudo
        spring_args=""
        export SUDO_ENVIRONMENT=0
    fi
    if [ "q${1}" = "qNO_SUDO" ];
    then
        # not sudo
        spring_args=""
        export SUDO_ENVIRONMENT=0
    fi
    
    ./apache-maven-3.8.6/bin/mvn spring-boot:run -Dspring-boot.run.arguments="$spring_args"
    \end{lstlisting}
    //TODO

\subsection{Микросервисы}
    Проект о каталогизации и структуризации данных (микросервисная архитектура):

    Идентификация микросервисов: Начало процесса с определения функциональных областей приложения, которые могут быть выделены в отдельные микросервисы.
    Проектирование API: Определение интерфейсов и методов взаимодействия между микросервисами, включая структуру запросов и ответов.
    Выбор технологий: Для каждого микросервиса выбираются технологии, наиболее подходящие для решения конкретных задач. Например, можно использовать Java с Spring для одного сервиса, а Python с Flask для другого.
    Разработка: Каждый микросервис разрабатывается отдельно, с фокусом на его конкретную функциональность и независимость от других сервисов.
    Тестирование и интеграция: После завершения разработки проводится тестирование каждого сервиса отдельно, а затем их интеграция для проверки взаимодействия и целостности системы.
    Развертывание и масштабирование: Каждый микросервис развертывается отдельно, что обеспечивает гибкость в масштабировании и управлении ресурсами. При необходимости можно масштабировать отдельные компоненты приложения независимо друг от друга.

    \begin{lstlisting}
    css
    
    microservices_project/
    │
    ├── service_1/
    │   ├── src/
    │   │   ├── controllers/
    │   │   │   ├── UserController.java
    │   │   │   └── ...
    │   │   ├── models/
    │   │   │   ├── User.java
    │   │   │   └── ...
    │   │   └── ...
    │   └── ...
    │
    ├── service_2/
    │   ├── src/
    │   │   ├── controllers/
    │   │   │   ├── AppointmentController.java
    │   │   │   └── ...
    │   │   ├── models/
    │   │   │   ├── Appointment.java
    │   │   │   └── ...
    │   │   └── ...
    │   └── ...
    │
    ├── service_3/
    │   ├── src/
    │   │   ├── controllers/
    │   │   │   ├── TransportationController.java
    │   │   │   └── ...
    │   │   ├── models/
    │   │   │   ├── Transportation.java
    │   │   │   └── ...
    │   │   └── ...
    │   └── ...
    │
    └── service_n/
        ├── src/
        │   ├── controllers/
        │   │   ├── OtherController.java
        │   │   └── ...
        │   ├── models/
        │   │   ├── OtherModel.java
        │   │   └── ...
        │   └── ...
        └── ...
    \end{lstlisting}
    
    \begin{lstlisting}
    sudo docker-compose build
    sudo docker-compose up
    \end{lstlisting}

    Проект состоит из 3 микросервисов и 2 отдельных служб:

    \subsubsection{Eureka}
        Микросервис, отвечающий за обнаружение служб, то есть позволяющий остальным микросервисам находить друг друга только по имени, не зная IP адресов и портов.

    \subsubsection{API Gateway}
        Микросервис, отвечающий за проброс портов. Позволяет организовать единую точку входа для всех микросервисов, переадресовывая пути по соответствующим микросервисам.

    \subsubsection{Service data}
        Микросервис, отвечающий за доступ к данным.

    \subsubsection{PSQL}
        База данных, хранящая данные.

    \subsubsection{Keycloak}
        Служба, осуществляющая авторизацию и регистрацию пользователей
    // TODO


% Обязательно добавляем это в конце каждой секции, чтобы 
% обеспечить переход на новую страницу
\clearpage