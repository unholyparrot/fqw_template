\anonsection{Основные понятия}

    \subsection*{Java}
        Java - строго типизированный объектно-ориентированный язык программирования общего назначения, разработанный компанией Sun Microsystems. Разработка ведётся сообществом, организованным через Java Community Process; язык и основные реализующие его технологии распространяются по лицензии GPL. Приложения Java обычно транслируются в специальный байт-код, поэтому они могут работать на любой компьютерной архитектуре, для которой существует реализация виртуальной Java-машины. Дата официального выпуска — 23 мая 1995 года. Занимает высокие места в рейтингах популярности языков программирования (2-е место в рейтингах IEEE Spectrum (2020) и TIOBE (2021)).

    \subsection*{Spring Framework}
        Spring Framework — универсальный фреймворк с открытым исходным кодом для Java-платформы. Фреймворк был впервые выпущен под лицензией Apache 2.0 license в июне 2003 года. Первая стабильная версия 1.0 была выпущена в марте 2004. Spring 2.0 был выпущен в октябре 2006, Spring 2.5 — в ноябре 2007, Spring 3.0 в декабре 2009, и Spring 3.1 в декабре 2011. Текущая версия — 5.3.x. Несмотря на то, что Spring не обеспечивал какую-либо конкретную модель программирования, он стал широко распространённым в Java-сообществе главным образом как альтернатива и замена модели Enterprise JavaBeans. Spring предоставляет бо́льшую свободу Java-разработчикам в проектировании; кроме того, он предоставляет хорошо документированные и лёгкие в использовании средства решения проблем, возникающих при создании приложений корпоративного масштаба.

    \subsection*{Keycloak}
        Keycloak - продукт с открытым кодом для реализации single sign-on с возможностью управления доступом, нацелен на современные приложения и сервисы. По состоянию на 2018 год, этот проект сообщества JBoss находится под управлением Red Hat которые используют его как upstream проект для своего продукта RH-SSO. Целью этого инструмента является сделать создание безопасных приложений и сервисов с минимальным написанием кода для аутентификации и авторизации. 

    \subsection*{Spring Cloud Netflix Eureka}
        Spring Cloud Netflix Eureka - инструмент для обнаружения служб на стороне клиента. Позволяет службам находить друг друга и связываться друг с другом без жесткого кодирования имени хоста и порта. Единственной «фиксированной точкой» в такой архитектуре является реестр служб, в котором должна регистрироваться каждая служба.

    \subsection*{Микросервисная архитектура}
        Микросервисная архитектура предполагает разработку и поддержку приложений с использованием небольших модульных сервисов, а не создание программного обеспечения в виде одного большого унифицированного блока кода (монолита). Основная концепция архитектуры в том, чтобы разделить сложное приложение на несколько небольших автономных и управляемых компонентов. Это позволяет повысить гибкость разработки, сократить time-to-market, улучшить отказоустойчивость и облегчить поддержку приложения.
        Каждый микросервис имеет собственный набор кода, базу данных и API для взаимодействия с другими сервисами. Они могут быть написаны на разных языках программирования и использовать различные технологии. Взаимодействие сервисов может осуществляться посредством сетевых запросов или сообщений.

    \subsection*{Монолитная архитектура архитектура}
        Монолитная архитектура — это традиционная модель программного обеспечения, которая представляет собой единый модуль, работающий автономно и независимо от других приложений. Монолитом часто называют нечто большое и неповоротливое, и эти два слова хорошо описывают монолитную архитектуру для проектирования ПО. Монолитная архитектура — это отдельная большая вычислительная сеть с единой базой кода, в которой объединены все бизнес-задачи.

% Обязательно добавляем это в конце каждой секции, чтобы 
% обеспечить переход на новую страницу
\clearpage