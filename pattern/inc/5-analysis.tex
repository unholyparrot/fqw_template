\section{Сравнение}

\subsection{Краткое содержание}
    Архитектурные особенности: Сравнение основных архитектурных принципов и подходов между монолитной и микросервисной архитектурами.
    Гибкость и масштабируемость: Оценка гибкости и возможностей масштабирования каждой из архитектур при изменении требований и нагрузки.
    Управление и развертывание: Сравнение процессов управления, развертывания и обновления приложений в контексте монолитной и микросервисной архитектур.
    Отказоустойчивость и производительность: Оценка уровня отказоустойчивости и производительности каждой архитектуры при различных условиях эксплуатации.
    Развитие и поддержка: Рассмотрение сложностей и преимуществ развития и поддержки приложений на основе монолитной и микросервисной архитектур.

    Сопоставление ключевых характеристик и результатов для монолитного и микросервисного подходов в контексте выбранных проектов:

    Архитектурные особенности:
        Монолитная архитектура: Единое приложение собрано в одном контейнере, что облегчает начальное развертывание и управление.
        Микросервисная архитектура: Приложение разделено на отдельные микросервисы, обеспечивая гибкость и независимость компонентов.

    Гибкость и масштабируемость:
        Монолитная архитектура: Ограниченная гибкость при масштабировании отдельных компонентов. Необходимость масштабировать всё приложение целиком.
        Микросервисная архитектура: Высокая гибкость в масштабировании и изменении отдельных компонентов без влияния на другие.

    Управление и развертывание:
        Монолитная архитектура: Проще в управлении и развертывании благодаря единому контейнеру, но требует перезапуска всего приложения при обновлении.
        Микросервисная архитектура: Более сложное управление из-за большего количества сервисов, но обновления могут быть более гибкими и безотказными.

    Отказоустойчивость и производительность:
        Монолитная архитектура: Отказ одной части может привести к отказу всего приложения. Масштабируемость ограничена.
        Микросервисная архитектура: Более высокая отказоустойчивость за счет изоляции отдельных сервисов. Возможность гибкого масштабирования отдельных компонентов для оптимизации производительности.

    Развитие и поддержка:
        Монолитная архитектура: Проще в разработке и поддержке для небольших и простых приложений с ограниченными требованиями.
        Микросервисная архитектура: Больше сложностей в разработке и поддержке из-за необходимости управления множеством сервисов, но более подходит для крупных и распределенных систем.

\subsection{Анализ производительности, масштабируемости, поддерживаемости и других аспектов каждой архитектуры}
    Монолитная архитектура:
    
        Производительность: Приложение может столкнуться с проблемами производительности из-за неэффективного масштабирования и обработки больших объемов данных в едином контейнере.
        Масштабируемость: Ограниченная масштабируемость из-за необходимости масштабировать всё приложение целиком, что может привести к избыточному использованию ресурсов.
        Поддерживаемость: Проще в поддержке из-за централизованной структуры, однако изменения в приложении могут потребовать пересборки и перезапуска всего приложения.
        Отказоустойчивость: Отказ одной части приложения может привести к полному отказу приложения, так как все компоненты работают в едином контексте.
    
    Микросервисная архитектура:
    
        Производительность: Более гибкая и эффективная производительность благодаря изоляции сервисов и возможности масштабирования только необходимых компонентов.
        Масштабируемость: Высокая масштабируемость, так как каждый сервис может быть масштабирован независимо от других, что позволяет оптимизировать использование ресурсов.
        Поддерживаемость: Более сложно в поддержке из-за необходимости управления множеством сервисов, но при правильном организации кода и процессов разработки это может быть облегчено.
        Отказоустойчивость: Более высокая отказоустойчивость за счет изоляции отдельных сервисов, что позволяет предотвратить полный отказ приложения из-за отказа одного компонента.

\subsection{Идентификация сильных и слабых сторон каждого подхода}
    Монолитная архитектура:

    Сильные стороны:
    
        Проще в разработке и начальном развертывании благодаря централизованной структуре.
        Меньше сложностей с управлением и развертыванием, так как всё приложение собрано в одном контейнере.
        Относительно более проста в поддержке для небольших и простых приложений с ограниченными требованиями.
    
    Слабые стороны:
    
        Ограниченная масштабируемость из-за необходимости масштабировать всё приложение целиком, что может привести к избыточному использованию ресурсов.
        Более сложно в управлении и поддержке при росте функциональности и увеличении объема данных.
        Больше риска полного отказа приложения из-за отказа одной из его частей.
    
    Микросервисная архитектура:
    
    Сильные стороны:
    
        Высокая гибкость и масштабируемость благодаря изоляции сервисов и возможности масштабирования только необходимых компонентов.
        Более высокая отказоустойчивость за счет изоляции отдельных сервисов, что позволяет предотвратить полный отказ приложения из-за отказа одного компонента.
        Более эффективное использование ресурсов при правильном масштабировании и настройке компонентов.
    
    Слабые стороны:
    
        Больше сложностей в управлении и развертывании из-за необходимости управления множеством сервисов и их зависимостей.
        Более высокие затраты на разработку и поддержку из-за необходимости управления множеством кодовых баз и интеграций между сервисами.
        Риск усложнения системы и возможности перехода к "монолитному" подходу при неправильном проектировании или неподходящем управлении.

\subsection{Идентификация сильных и слабых сторон каждого подхода}
    Лучше
    //TODO

\subsection{Идентификация сильных и слабых сторон каждого подхода}
    Лучше
    //TODO
    
        
% Обязательно добавляем это в конце каждой секции, чтобы 
% обеспечить переход на новую страницу
\clearpage