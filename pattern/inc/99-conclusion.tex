\anonsection{Заключение}

    Целью данного исследования был сравнительный анализ монолитного и микросервисного подходов к разработке веб-приложений в условиях низкой загрузки. Исследование направлено на выявление преимуществ и недостатков каждой архитектуры, чтобы предоставить рекомендации по выбору подходящей архитектуры для малых и средних проектов.

    В ходе работы было представлено комплексное описание монолитной и микросервисной архитектур, включающее в себя несколько пунктов. Были проанализированы уже существующие исследования с целью понимания текущих знаний по данной теме и подтверждения новизны данной работы. Также были подробно описаны два проекта, разработанные командой под моим руководством, имеющие разную архитектуру, на примере которых и проводился сравнительный анализ. Наконец, были рассмотрены ключевые вопросы, связанные с различными аспектами разработки и эксплуатации веб-приложений, включая такие критерии, как команда и скорость разработки, совместимость, переносимость, наблюдаемость и потребление ресурсов. Эти критерии были выбраны для того, чтобы дать всестороннюю оценку эффективности и применимости каждой архитектуры, сфокусировавшись на условиях низкой загрузки.
    
    Команда и скорость разработки: Монолитная архитектура часто способствует более быстрой разработке, так как вся команда работает над одной кодовой базой. Это упрощает коммуникацию и координацию, что может быть критически важным для малых команд.
    
    Совместимость: Несмотря на то, что каждый микросервис можно адаптировать для взаимодействия с различными системами и сервисами, монолитная архитектура обеспеичвает прямое, непосредственное взаимодействие, что важнее для упрощения разработки.
    
    Переносимость: Микросервисы обеспечивают лучшую переносимость, так как каждый сервис можно развернуть и масштабировать независимо на разных платформах и в различных средах, используя контейнеризацию и оркестрацию. Тем не менее, контейнеризация применима так же и к монолитнной архитектуре, что может уменьшить риски и повысить удобство управления приложением.
    
    Наблюдаемость: Монолитная архитектура позволяет естественным образом аггрегировать логи, а также проихводить отладку с помощью одного специализированного приложения в едином месте, что упрощает разработку. Тогда как микросервисная архитектура требует дополнительных усилий для настройки и управления мониторингом.
    
    Потребление ресурсов: В условиях низкой загрузки монолитная архитектура часто оказывается более эффективной, так как отсутствуют накладные расходы на межсервисные коммуникации. Однако микросервисы могут быть лучше оптимизированы в долгосрочной перспективе благодаря возможности независимого масштабирования.
    
    Для малых и средних проектов с низкой загрузкой рекомендуется тщательно оценить текущие и будущие потребности проекта. Монолитная архитектура может быть предпочтительнее в случаях, когда важны скорость разработки и простота управления. Микросервисная архитектура может подойти, если проект планируется масштабировать и интегрировать с другими системами в будущем.
    
    Монолитная архитектура оптимальна для проектов с ограниченными ресурсами и небольшими командами разработчиков, где важны скорость разработки и простота управления. Она также подходит для проектов, которые не предполагают значительного роста или масштабирования в ближайшем будущем. Для стартапов и прототипов монолитный подход может быть идеальным, позволяя быстро вывести продукт на рынок. Микросервисная архитектура рекомендуется для проектов, где требуется высокая гибкость и возможность масштабирования отдельных компонентов. Она также подходит для систем, которые требуют частых обновлений и интеграции с внешними сервисами. Проекты, которые предполагают распределенную разработку с участием нескольких команд, также могут выиграть от использования микросервисного подхода.
    
    При выборе архитектуры важно учитывать не только текущие требования проекта, но и его долгосрочные цели. Если проект предполагает быстрый рост и развитие, микросервисы могут быть более подходящими. Для проектов с четко определенными и стабильными требованиями монолитная архитектура может оказаться более эффективной. Важно также учитывать навыки команды и наличие ресурсов для поддержки выбранной архитектуры.
    
    Будущие исследования могут быть направлены на изучение монолитных и микросервисных архитектур в условиях высокой загрузки, чтобы оценить их производительность и масштабируемость в более требовательных сценариях. Также целесообразно исследовать влияние новых технологий и инструментов на выбор архитектуры. Наконец, важно рассмотреть гибридные подходы, которые могут объединять преимущества обеих архитектур, для чего можно разработать третий проект и провести аналогичный анализ.

% Обязательно добавляем это в конце каждой секции, чтобы 
% обеспечить переход на новую страницу
\clearpage