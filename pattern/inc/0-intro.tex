\anonsection{Введение}
    В современном мире веб-разработка стала неотъемлемой частью программирования. Согласно статистике Гитхаб \cite{githut}, более 15\% звёзд принадлежат репозиториям, использующим язык Java Script, что однозначно показывает популярность веб разработки. При этом, согласно статистике сайта siteefy \cite{websitesstats}, более 250000 сайтов создаются каждый день - следовательно, популярна не только поддержка старых репозиториев веб разработки, но и разработка новых сайтов.
    
    Выбор архитектурного подхода при создании веб-приложений играет решающую роль в обеспечении их эффективности, масштабируемости и удобства использования. Не случайно вопрос выбора между монолитной и микросервисной архитектурами остается одним из наиболее актуальных в сфере веб-разработки. Выбор архитектурного подхода определяет не только технические характеристики разрабатываемого приложения, но и его конечное качество и степень соответствия потребностям пользователей; это связано с тем, что архитектура веб-приложения влияет на его производительность, масштабируемость, надежность, безопасность, и многие другие аспекты, определяющие его успешность на рынке.
    
    Несмотря на появление новых подходов и технологий, выбор между монолитной и микросервисной архитектурами всё ещё остается актуальным. Монолитная архитектура - традиционный способ построения приложения, при котором весь функционал собирается в одном приложении или модуле, и все компоненты, функции, классы взаимодействуют друг с другом напрямую а рамках одного процесса. Микросервисная архитектура, в противовес монолитной - современный подход, использующий разделение логически разных частей приложения на физически разные компоненты - микросервисы. Каждый микросервис разрабатывается и разворачивается независимо, общаются между собой чаще всего через http запросы. Монолитные приложения сохраняют свою популярность благодаря своей простоте и относительной легкости в развертывании, в то время как микросервисные архитектуры привлекают внимание своей гибкостью и масштабируемостью.
    
    В данной работе мы сфокусируемся на  на анализе нефункциональных требований в контексте выбора архитектуры для веб-приложений. В частности, мы рассмотрим влияние выбора между монолитной и микросервисной архитектурами на безопасность, производительность, масштабируемость и другие аспекты разработки веб-приложений.

% Обязательно добавляем это в конце каждой секции, чтобы 
% обеспечить переход на новую страницу
\clearpage