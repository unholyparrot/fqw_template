\section{Анализ предметной области}

\subsection{Описание проблемы}
    В данном разделе опишем проблемы, решаемые при сравнении монолитного и микросервисного подходов к разработке веб-приложений в условиях низкой загрузки.
    Три ключевых аспекта, которые следует учитывать при сравнении монолитного и микросервисного подходов, следующие:
    \begin{itemize}
        \item основной фокус ставится на архитектурные особенности и влияние на бизнес-логику;
        \item сложность и масштабируемость систем зависят от выбранной архитектуры;
        \item взаимодействие и сотрудничество между техническими специалистами различных областей.
    \end{itemize}
    Последний пункт фактически является самым важным для успешного внедрения и эксплуатации выбранного архитектурного подхода. Именно он позволяет при решении задачи максимально эффективно использовать ресурсы и достигать поставленных целей.

    Монолитный подход, вопреки распространённому мнению, не является устаревшим или неэффективным. Он также имеет определённые преимущества, такие как упрощенное управление и развертывание, что особенно важно в условиях низкой загрузки. Однако, с увеличением сложности и размеров приложений, монолитная архитектура может становиться менее гибкой и более сложной для масштабирования. Микросервисный подход, в свою очередь, предлагает множество преимуществ, таких как независимое развертывание и возможность использования различных технологий для разных сервисов. Это позволяет легче управлять сложными системами и масштабировать их по мере необходимости. Тем не менее, микросервисная архитектура может быть избыточной и сложной в условиях низкой загрузки, требуя дополнительных ресурсов для управления и оркестрации сервисов.
    
    Разработка качественного веб-приложения зависит от множества внешних и внутренних факторов, таких как бюджет, опыт команды, слаженность команды, квалифицированность мэнеджеров, требования заказчика. Тем не менее, путём выбора подходящей архитектуры возможно устранить большую часть подобных проблем.

\subsection{Акутальность}
    
    Данные рассуждения уже показывают важность первой части темы данного диплома: "сравнительный анализ монолитного и микросервисного подходов к разработке web приложений...". Остаётся вопрос - почему была выбрана именно низкая загрузка? Выбор сравнительного анализа монолитного и микросервисного подходов к разработке веб-приложений в условиях низкой загрузки был обусловлен несколькими важными факторами. Во-первых, в современных условиях многие компании, особенно стартапы и малые предприятия, начинают свою деятельность с небольших масштабов. Для них важна оптимизация ресурсов и минимизация затрат при сохранении гибкости и возможности роста. Поэтому понимание того, какая архитектура наиболее эффективна при низкой загрузке, имеет практическую значимость.

    Во-вторых, условия низкой загрузки предоставляют уникальную возможность для анализа эффективности и управляемости архитектурных подходов без влияния факторов, связанных с масштабированием под высокой нагрузкой. В таких условиях можно более детально рассмотреть преимущества и недостатки монолитной и микросервисной архитектур, выявить ключевые аспекты, которые могут повлиять на выбор архитектурного стиля в долгосрочной перспективе. Это также позволяет лучше понять, как каждый подход справляется с изменениями и поддержкой на начальных этапах развития проекта.

    Более того, данное исследование фокусируется на сравнении двух конкретных примеров проектов (подробнее в разделе Реализация), которые были реализованы единой командой в примерно одинаковые сроки, по непосредственным требованиям заказчика. В начале обоих проектов мы выбирали архитектуру, и оба выбора существенно повлияли на характеристики проекта, трудозатраты и скорость разработки. Таким образом, данное исследование применимо и необходимо в современной разработке web приложений.

\subsection{Существующие исследования}
    Для проведения анализа существующих исследований были использованы такие интернет ресурсы, как Google Scholar и Электронная библиотека Физтеха. В результате формируется общая картина: существующие исследования уже обратили внимание на различия между монолитной и микросервисной архитектурами, но в основном они фокусируются на условиях высокой загрузки и масштабируемости. Например, исследование “Monolithic vs. Microservice Architecture: A Performance and Scalability Evaluation” \cite{blinowski2022monolithic} проводит детальный анализ масштабируемости обеих архитектур. В этом исследовании рассматриваются различные сценарии высокой нагрузки и анализируется, как каждая архитектура справляется с увеличением числа запросов и данных. Результаты показывают, что микросервисная архитектура предлагает значительные преимущества в условиях высокой нагрузки благодаря своей гибкости и способности масштабироваться горизонтально.

    Еще одно важное исследование, “A Comparative Review of Microservices and Monolithic Architectures” \cite{al2018comparative}, уделяет внимание производительности при распараллеливании. Исследование оценивает, как обе архитектуры справляются с задачами, требующими параллельного выполнения операций. Результаты показывают, что микросервисы могут более эффективно распределять нагрузку между различными сервисами, что позволяет им лучше справляться с задачами, требующими параллельной обработки. Однако это исследование также концентрируется на условиях, где нагрузка на систему значительно выше, чем в случае низкой загрузки.
    
    Третье исследование, “Migrating from monolithic architecture to microservices: A Rapid Review” \cite{ponce2019migrating}, фокусируется на процессах миграции от монолитной архитектуры к микросервисам. В этом исследовании рассматриваются различные стратегии и подходы к миграции, анализируются преимущества и недостатки каждого метода, а также их влияние на общую производительность и управляемость системы. Исследование предоставляет ценные инсайты для организаций, планирующих переход на микросервисную архитектуру, но опять же, оно больше ориентировано на системы, испытывающие значительные нагрузки и требующие повышения масштабируемости.
    
    Несмотря на ценность этих исследований, они в основном сосредоточены на условиях высокой нагрузки и не акцентируют внимание на сценарии, когда система работает при низкой загрузке. В моем исследовании основной акцент делается на сравнении монолитного и микросервисного подходов в условиях низкой загрузки. Это позволяет выявить, как каждая архитектура справляется с задачами в условиях, где производительность и масштабируемость не являются первоочередными факторами, а важнее простота управления и экономия ресурсов.
    
    Мое исследование также основано на реальных примерах двух проектов, разработанных мной и моей командой. Оба проекта имеют примерно одинаковую сложность, что позволяет провести объективный и детализированный анализ. Это отличается от существующих исследований тем, что мы исследуем эффективность и целесообразность каждого подхода в контексте начальных стадий разработки и низкой загрузки, что предоставляет уникальную перспективу и ценные рекомендации для стартапов и малых предприятий, начинающих свою деятельность.

\subsection{Особенности каждого подхода}
    Опишем основные принципы и характеристики монолитной подхода и микросервисной архитектур.
    Монолитный подход:
    \begin{itemize}
        \item единый исполняемый контейнер: все компоненты приложения размещаются в одном исполняемом контейнере;
        \item монолитные приложения обычно написаны на одном языке программирования и используют одну базу данных;
        \item простота развертывания и управления - так как все компоненты находятся в одном приложении, развертывание и управление приложением обычно проще;
        \item ограничения в масштабировании - монолитные приложения могут столкнуться с проблемами масштабирования при увеличении объема функциональности.
    \end{itemize}

    Микросервисный подход:
     \begin{itemize}
        \item разделение на небольшие сервисы: приложение разбивается на независимые компоненты, каждый из которых выполняет определенную функциональность;
        \item гибкость в выборе технологий: каждый сервис может быть написан на разных языках программирования и использовать различные технологии;
        \item распределенная архитектура - каждый сервис может иметь свою собственную базу данных и взаимодействует с другими сервисами через API;
        \item масштабируемость и гибкость - микросервисы позволяют гибко масштабировать отдельные компоненты приложения в зависимости от нагрузки.
    \end{itemize}

    Оба подхода имеют свои преимущества и недостатки, и выбор между ними зависит от конкретных требований проекта, его масштаба и потребностей в гибкости и масштабируемости. Преимущества монолитной архитектуры заключаются в простоте разработки (единое приложение облегчает разработку и тестирование), простоте в управлении (единое приложение упрощает развертывание и мониторинг приложения), уменьшении накладных расходов (отсутствие сетевых запросов между компонентами приложения снижает накладные расходы на коммуникацию). Основные недостатки включают в себя сложности в масштабировании (монолитные приложения могут столкнуться с проблемами масштабирования при увеличении объема функциональности), затруднения в обновлении (изменения в одной части приложения могут повлиять на всю систему и требовать полного пересборки и перезапуска) и ограниченную гибкость (сложнее добавлять новые функции и технологии из-за зависимостей между компонентами).
    
    Микросервисная архитектура обладает практически противоположными характеристиками: гибкость и масштабируемость (каждый сервис может масштабироваться независимо, что обеспечивает гибкость в управлении ресурсами), улучшенная отказоустойчивость (отказ одного сервиса не приводит к полной недоступности приложения), "технологическая свобода" (различные сервисы могут быть реализованы на разных технологиях, что позволяет использовать наиболее подходящие инструменты для каждой задачи). Из недостатков стоит выделить сложности в развертывании и управлении (управление распределенными системами требует дополнительных усилий и навыков), увеличение сложности тестирования (необходимость тестирования взаимодействия между сервисами увеличивает сложность тестирования, несмотря на упрощение unit тестирование ввиду логической разделённости сервисов) и дополнительные затраты на сетевое взаимодействие (сетевые запросы между сервисами могут вызывать задержки и увеличивать нагрузку на сеть).

    Ниже представлена таблица 1.1, подводящая итог рассуждениям выше

    \begin{table}
        \centering
        \begin{tabular}{ p{3cm} p{5cm} p{5cm} }
             & Микросервисы & Монолит \\ \hline
            сложность разработки & простая & сложная \\ \hline
            сложность управления & простое & сложное \\ \hline
            количество накладных расходов & небольшое & большое \\ \hline
            масштабирование & сложное & лёгкое \\ \hline
            надёжность & хрупкая & надёжная \\ \hline
            гибкость & меньше & больше \\
        \end{tabular}
        \caption{Теоретическое сравнение архитектур}
        \label{tab:my_label}
    \end{table}

\subsection{Гипотеза}
    ???TODO???
        
% Обязательно добавляем это в конце каждой секции, чтобы 
% обеспечить переход на новую страницу
\clearpage