\section{Анализ предметной области}

\subsection{Описание проблемы}
    //TODO
    https://colorlib.com/wp/web-development-statistics/

\subsection{Описание основных принципов и особенностей каждого подхода}
        Монолитный подход:
        Единый исполняемый контейнер: все компоненты приложения размещаются в одном исполняемом контейнере.
        Монолитные приложения обычно написаны на одном языке программирования и используют одну базу данных.
        Простота развертывания и управления: так как все компоненты находятся в одном приложении, развертывание и управление приложением обычно проще.
        Ограничения в масштабировании: монолитные приложения могут столкнуться с проблемами масштабирования при увеличении объема функциональности.

    Микросервисный подход:
        Разделение на небольшие сервисы: приложение разбивается на независимые компоненты, каждый из которых выполняет определенную функциональность.
        Гибкость в выборе технологий: каждый сервис может быть написан на разных языках программирования и использовать различные технологии.
        Распределенная архитектура: каждый сервис может иметь свою собственную базу данных и взаимодействует с другими сервисами через API.
        Масштабируемость и гибкость: микросервисы позволяют гибко масштабировать отдельные компоненты приложения в зависимости от нагрузки.
        
    Оба подхода имеют свои преимущества и недостатки, и выбор между ними зависит от конкретных требований проекта, его масштаба и потребностей в гибкости и масштабируемости.

    Преимущества и недостатки монолитной и микросервисной архитектур:

    Монолитная архитектура:
    
    Преимущества:
    
        Простота разработки: единое приложение облегчает разработку и тестирование.
        Проще в управлении: единый контейнер упрощает развертывание и мониторинг приложения.
        Меньшие накладные расходы: отсутствие сетевых запросов между компонентами приложения снижает накладные расходы на коммуникацию.
    
    Недостатки:
    
        Сложности в масштабировании: монолитные приложения могут столкнуться с проблемами масштабирования при увеличении объема функциональности.
        Затруднения в обновлении: изменения в одной части приложения могут повлиять на всю систему и требовать полного пересборки и перезапуска.
        Ограниченная гибкость: сложнее добавлять новые функции и технологии из-за зависимостей между компонентами.
    
    Микросервисная архитектура:
    
    Преимущества:
    
        Гибкость и масштабируемость: каждый сервис может масштабироваться независимо, что обеспечивает гибкость в управлении ресурсами.
        Улучшенная отказоустойчивость: отказ одного сервиса не приводит к полной недоступности приложения.
        Технологическая свобода: различные сервисы могут быть реализованы на разных технологиях, что позволяет использовать наиболее подходящие инструменты для каждой задачи.
    
    Недостатки:
    
        Сложности в развертывании и управлении: управление распределенными системами требует дополнительных усилий и навыков.
        Увеличение сложности тестирования: необходимость тестирования взаимодействия между сервисами увеличивает сложность тестирования.
        Дополнительные затраты на сетевое взаимодействие: сетевые запросы между сервисами могут вызывать задержки и увеличивать нагрузку на сеть.


\subsection{Аналогичные исследования}
    //TODO
    
        
% Обязательно добавляем это в конце каждой секции, чтобы 
% обеспечить переход на новую страницу
\clearpage